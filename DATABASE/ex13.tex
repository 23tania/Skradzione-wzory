\documentclass{article}
\usepackage[utf8]{inputenc}

\begin{document}
\title{Liczby zespolone}
\maketitle

\section{Liczby zespolone}
Liczby zespolone – liczby będące elementami rozszerzenia ciała liczb rzeczywistych o jednostkę urojoną i , , to znaczy pierwiastek wielomianu x 2 + 1. Liczby zespolone rozszerzają koncepcję jednowymiarowej osi liczbowej do dwuwymiarowej płaszczyzny zespolonej, przy zastosowaniu osi poziomej do oznaczenia liczb rzeczywistych, a pionowej do oznaczenia liczb urojonych. Liczba zespolona postaci a + b i  może być określona za pomocą współrzędnych ( a , b ) na płaszczyźnie zespolonej.

Liczby zespolone pozbawione części rzeczywistej, a zatem leżące bezpośrednio na osi pionowej płaszczyzny zespolonej, nazywane są liczbami urojonymi, zaś liczby pozbawione części urojonej, a więc leżące bezpośrednio na osi poziomej, to liczby rzeczywiste. Zbiór liczb zespolonych zawiera zatem w sobie zbiór liczb rzeczywistych, rozszerzony w celu umożliwienia rozwiązywania takich problemów, które nie posiadają rozwiązania w zbiorze liczb rzeczywistych. Poza matematyką liczby zespolone znajdują zastosowanie także w innych dziedzinach nauki, jak fizyka, chemia, biologia, ekonomia, elektrotechnika i statystyka.

Po raz pierwszy pojęcie liczb zespolonych, jako składających się z części rzeczywistej oraz urojonej, wprowadził niemiecki matematyk Carl Friedrich Gauss w 1832[1]. Problem istnienia pól o ujemnej wartości rozważał znacznie wcześniej włoski matematyk Girolamo Cardano podczas prób znalezienia rozwiązań równań sześciennych w XVI wieku. Nazywał je liczbami fikcyjnymi[2]. Kartezjusz nadał im nazwę liczb urojonych w pracy wydanej w 1637[3]. Samo istnienie pierwiastka kwadratowego liczby ujemnej było najprawdopodobniej po raz pierwszy rozważane już w starożytności przez Herona z Aleksandrii[1]. 
\section{Działania}
Dodawanie, odejmowanie i mnożenie liczb zespolonych w postaci algebraicznej wykonuje się tak samo jak odpowiednie operacje na wyrażeniach algebraicznych, przy czym i 2 = − 1 :
$$(a+bi)\pm (c+di)=(a\pm c)+(b\pm d)i$$
$${ (a+bi)(c+di)=ac+(bc+ad)i+bdi^{2}=(ac-bd)+(bc+ad)i.}$$
Aby podzielić przez siebie dwie liczby zespolone, wystarczy pomnożyć dzielną i dzielnik przez liczbę sprzężoną do dzielnika (analogicznie do usuwania niewymierności z mianownika w wyrażeniach algebraicznych): 
$${ {\frac {a+bi}{c+di}}={\frac {(a+bi)(c-di)}{(c+di)(c-di)}}={\frac {(ac+bd)+(bc-ad)i}{c^{2}+d^{2}}}.}$$

\section{Płaszczyzna zespolona}
Liczbom zespolonym można przyporządkować wzajemnie jednoznacznie wektory na płaszczyźnie, podobnie jak utożsamia się wektory na prostej z liczbami rzeczywistymi (w obu przypadkach można utożsamiać również same punkty, gdyż wspomniane wektory zaczepia się w początku układów współrzędnych).

Każdej więc liczbie zespolonej z = a + b i można przyporządkować wektor z → = ( a , b ) i odwrotnie. Działania dodawania i mnożenia w liczbach zespolonych odpowiadają następującym działaniom na wektorach: 
$${ (a,b)+(c,d)=(a+c,b+d),}$$
$${ (a,b)(c,d)=(ac-bd,bc+ad).}$$
Liczba zespolona może być zatem wyrażona przez długość jej wektora (moduł) oraz jego kąt skierowany (argument): 
$${ z=a+bi=|z|{\tfrac {a}{|z|}}+|z|{\tfrac {b}{|z|}}i=|z|(\cos \varphi +i\sin \varphi ).}$$
Powyższą postać liczby zespolonej nazywa się postacią trygonometryczną (z powodu użycia funkcji trygonometrycznych), biegunową (jest przedstawieniem liczby zespolonej we współrzędnych biegunowych) lub geometryczną (prowadzi do geometrycznej interpretacji liczb zespolonych na płaszczyźnie). Warto zauważyć, że postać algebraiczna odpowiada współrzędnym prostokątnym.

Liczby zespolone w postaci trygonometrycznej są równe, gdy ich moduły i argumenty są równe, tj. z = a + b i  oraz u = c + d i  u=c+di są równe, gdy 
$$|z|={\sqrt  {a^{2}+b^{2}}}={\sqrt  {c^{2}+d^{2}}}=|u|$$
$${ \operatorname {Arg} \;z=\operatorname {Arg} \;u.}$$
Wzory pozwalające na przejście od postaci trygonometrycznej do algebraicznej są oczywiste: 
$${ {\begin{cases}a=|z|\cos \varphi \\b=|z|\sin \varphi \end{cases}}.}$$
Warto zwrócić uwagę na mnożenie liczb w postaci trygonometrycznej, niech 
$${ x=|x|(\cos \alpha +i\sin \alpha ),}$$
\end{document}