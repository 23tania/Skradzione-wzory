\documentclass{article}
\usepackage[utf8]{inputenc}
\usepackage[MeX,plmath]{polski} 
\usepackage{amsmath}
\usepackage{amsfonts}

\begin{document}
\title{Równania Hamiltona}
\maketitle

\section*{Wprowadzenie}
Równania Hamiltona – jedna z alternatywnych postaci zapisu równań ruchu, obok równań ruchu mechaniki Newtona oraz równań Eulera-Lagrange’a mechaniki w ujęciu Lagrange’a. Równania te wyrażają pochodne współrzędnych uogólnionych i pędów uogólnionych układu fizycznego po czasie przy pomocy funkcji Hamiltona układu. 

\section*{Definicja równań Hamiltona}
Równania Hamiltona – układ równań opisujących zmiany w czasie współrzędnych uogólnionych i pędów uogólnionych układu fizycznego wyrażonych przy pomocy funkcji Hamiltona 
\begin{displaymath}
\left\{{\begin{matrix}{\dot {p}}_{i}=-{\cfrac {\partial H}{\partial q_{i}}}\\{[.5em]}{\dot {q}}_{i}={\cfrac {\partial H}{\partial p_{i}}}\end{matrix}}\right.\quad i=1,\dots ,s
\end{displaymath}

Równania Hamiltona stanowią układ 2s równań różniczkowych zwyczajnych pierwszego rzędu. 

\section*{Równania Hamiltona wyrażone przez nawiasy Poissona}
Przy zapisie z użyciem nawiasów Poissona układ ten wygląda bardziej symetrycznie 
\begin{equation*}
\left\{{\begin{matrix}{\dot {p}}_{i}=\{p_{i},H\}\\{[.5em]}{\dot {q}}_{i}=\{q_{i},H\}\end{matrix}}\right.\quad i=1,\dots ,s
\end{equation*}

\section*{Twierdzenie}
Jeżeli układ fizyczny znajduje się w polu oddziaływań o potencjale skalarnym, np. ciecz porusza się w polu grawitacyjnym, to pęd cząstek układu jest proporcjonalny do ich prędkości. Ponadto jeżeli równania ruchu cząstek cieczy są równaniami Hamiltona, to ciecz ta jest nieściśliwa, tzn. jej super-prędkość ma znikającą dywergencję
$$ \nabla \cdot ({\dot {\mathbf {q} }},{\dot {\mathbf {p} }})=0 $$
gdzie:
\[ \nabla \cdot ({\dot {\mathbf {q} }},{\dot {\mathbf {p} }})=\sum _{i}\left({\frac {\partial {\dot {q_{i}}}}{\partial q_{i}}}+{\frac {\partial {\dot {p_{i}}}}{\partial p_{i}}}\right) \]

Zakładając, na wzór elektrodynamiki, istnienie skalarnego potencjału ,,wektorowego'' H, którego odpowiednik rotacji, jak permutacja gradientu z sygnaturą (jeden z wektorów prostopadłych do wektora całkowitego gradientu Hamiltonianu), zagwarantuje znikanie dywergencji podobnie jak w 3 wymiarach dla pól elektromagnetycznych, tzn. takiego, że 
$$ {\dot {p}}_{i}=-{\frac {\partial H}{\partial q_{i}}} $$
$$ {\dot {q}}_{i}={\frac {\partial H}{\partial p_{i}}} $$
otrzymujemy z twierdzenia Schwartza o przemienności pochodnych cząstkowych
\[ \nabla \cdot ({\dot {\mathbf {q} }},{\dot {\mathbf {p} }})=\sum _{i}\left({\frac {\partial ^{2}H}{\partial q_{i}\partial p_{i}}}-{\frac {\partial ^{2}H}{\partial p_{i}\partial q_{i}}}\right)=0 \]

Jak widać także
\[ \nabla H\cdot \not \nabla H=\sum _{i}\left({\frac {\partial H}{\partial q_{i}}}{\frac {\partial H}{\partial p_{i}}}-{\frac {\partial H}{\partial p_{i}}}{\frac {\partial H}{\partial q_{i}}}\right)=0 \]
jeśli zapiszemy równania Hamiltona symbolicznie w sposób skrócony: 
$$ ({\dot {\mathbf {q} }},{\dot {\mathbf {p} }})=\not \nabla H $$
co wyraża prostopadłość wektora prawej strony równań do gradientu Hamiltonianu H. 

\section*{Przykład - oscylator harmoniczny}
Hamiltonian jednowymiarowego oscylatora harmonicznego o jednostkowej masie i częstości dany jest przez: 
\begin{displaymath}
H={\frac {p^{2}}{2}}+{\frac {x^{2}}{2}}
\end{displaymath}

\end{document}