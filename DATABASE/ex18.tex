\documentclass{article}
\usepackage[utf8]{inputenc}
\usepackage[MeX,plmath]{polski} 
\usepackage{amsmath}
\usepackage{amsfonts}

\begin{document}
\title{Rozkład normalny}
\maketitle

\section*{Wprowadzenie}
Rozkład normalny, rozkład Gaussa (w literaturze francuskiej zwany rozkładem Laplace’a-Gaussa) – jeden z najważniejszych rozkładów prawdopodobieństwa, odgrywający ważną rolę w statystyce. Wykres funkcji prawdopodobieństwa tego rozkładu jest krzywą w kształcie dzwonu (tak zwaną krzywą dzwonową).

Przyczyną jego znaczenia jest częstość występowania w naturze. Jeśli jakaś wielkość jest sumą lub średnią bardzo wielu drobnych losowych czynników, to niezależnie od rozkładu każdego z tych czynników jej rozkład będzie zbliżony do normalnego (centralne twierdzenie graniczne) – dlatego można go bardzo często zaobserwować w danych. Ponadto rozkład normalny ma interesujące właściwości matematyczne, dzięki którym oparte na nim metody statystyczne są proste obliczeniowo. 

\section*{Definicja rozkładu normalnego}
Istnieje wiele równoważnych sposobów zdefiniowania rozkładu normalnego. Należą do nich: funkcja gęstości, dystrybuanta, momenty, kumulanty, funkcja charakterystyczna, funkcja tworząca momenty i funkcja tworząca kumulanty. Wszystkie kumulanty rozkładu normalnego wynoszą 0 oprócz pierwszych dwóch. 
\subsection*{Funkcja gęstości}
Funkcja gęstości prawdopodobieństwa rozkładu normalnego jest przykładem funkcji Gaussa. Dana jest ona wzorem: 
$$ f_{\mu ,\sigma }(x)={\frac {1}{\sigma {\sqrt {2\pi }}}}\,\exp \left({\frac {-(x-\mu )^{2}}{2\sigma ^{2}}}\right) $$

Jeśli \( \mu =0 \) i $ \sigma =1$, to rozkład ten nazywa się standardowym rozkładem normalnym, jego funkcja gęstości opisana jest wzorem: 
\begin{displaymath}
\phi _{0,1}(x)=\phi (x)={\frac {1}{\sqrt {2\pi }}}\,\exp \left(-{\frac {x^{2}}{2}}\right)
\end{displaymath}

\subsection*{Dystrybuanta}
Dystrybuanta jest definiowana jako prawdopodobieństwo tego, że zmienna X ma wartości mniejsze bądź równe x i w kategoriach funkcji gęstości wyrażana jest (dla rozkładu normalnego) wzorem: 
\begin{equation*}
P(X \leq x)=\int \limits _{-\infty }^{x}{\frac {1}{\sigma {\sqrt {2\pi }}}}e^{\frac {-(x-\mu )^{2}}{2\sigma ^{2}}}\,dx
\end{equation*}

Całki powyższej nie da się obliczyć dokładnie metodą analityczną. W konkretnych zagadnieniach do obliczenia wartości dystrybuanty stosuje się zatem tablice statystyczne (bądź też odpowiednie kalkulatory czy oprogramowanie komputerów)
\begin{gather*}
\Phi (z)=\int \limits _{-\infty }^{z}{\frac {1}{\sqrt {2\pi }}}\,e^{-{\frac {t^{2}}{2}}}\,dt
\end{gather*}

Związek dystrybuanty i dystrybuanty rozkładu normalnego o dowolnie zadanych parametrach otrzymuje się za pomocą standaryzowania rozkładu. 
$$ P(X\leq x)=\Phi \left({\frac {x-\mu }{\sigma }}\right) $$

Dystrybuanta standardowego rozkładu normalnego może być wyrażona poprzez funkcję specjalną (nieelementarną, przestępną), tzw. funkcję błędu jako: 
\begin{displaymath}
\Phi (z)={\frac {1}{2}}\left(1+\operatorname {erf} \,{\frac {z}{\sqrt {2}}}\right)
\end{displaymath}

\subsection*{Funkcje tworzące}
Funkcją charakterystyczną rozkładu normalnego jest 
\begin{equation*}
\varphi (t)=\exp \left(i\mu t-{\frac {\sigma ^{2}t^{2}}{2}}\right)
\end{equation*}

W przypadku standardowego rozkładu normalnego ma ona postać:
$$ \varphi (t)=\exp \left(-{\frac {t^{2}}{2}}\right) $$

\end{document}