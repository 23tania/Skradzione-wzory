\documentclass{article}
\usepackage[utf8]{inputenc}

\begin{document}
\title{Równanie Schrödingera}
\maketitle

\section{Równanie Schrödingera}
Równanie Schrödingera – jedno z podstawowych równań nierelatywistycznej mechaniki kwantowej (obok równania Heisenberga), sformułowane przez austriackiego fizyka Erwina Schrödingera w 1926 roku. Równanie to pozwala opisać ewolucję stanu układu kwantowego w czasie w sposób znacznie dokładniejszy, niż czyni to mechanika klasyczna.

W nierelatywistycznej mechanice kwantowej równanie Schrödingera odgrywa rolę fundamentalną, analogiczną do roli zasad dynamiki Newtona w mechanice klasycznej[1]. 

${ {\hat {H}}{\big |}\Psi (t)\rangle =i\hbar {\frac {\partial }{\partial t}}{\big |}\Psi (t)\rangle ,}$


\section{Reprezentacja położeniowa}
Reprezentację położeniową wybiera się, gdy trzeba rozwiązać problem ruchu cząstek w przestrzeni. W tej reprezentacji równanie Schrödingera przyjmuje postać: 
${ {\hat {H}}\Psi (r,t)=i\hbar {\frac {\partial }{\partial t}}\Psi (r,t),}$

\section{Reprezentacja spinowa}
Gdy trzeba znaleźć zmiany czasowe stanów spinowych cząstek, to przyjmuje się reprezentację spinową; hamiltonian nie ma tu postaci pojedynczego operatora, ale jest operatorem o postaci macierzowej. Przykładowo, dla pojedynczej cząstki o spinie 1/2 hamiltonian ma postać macierzy 2x2
Np. w przypadku elektronu znajdującym się w zewnętrznym polu  magnetycznym część operatora Hamiltona odpowiadająca energii oddziaływania elektronu z polem magnetycznym ma postać 
\begin{math}
{ {\hat {H}}=-\mu {\vec {B}}\cdot {\vec {\sigma }},}
\end{math}

\section{Równanie Schrödingera niezależne od czasu}
Jeżeli układ fizyczny oddziałuje z otoczeniem, to operator Hamiltona jest wyrażony przez pochodne względem r.
\\Mówi się, że hamiltonian zależy od czasu. Wtedy znalezienie opisu stanu układu kwantowego wymaga stosowania ogólnego równania Schrödingera.

Sytuacja upraszcza się, gdy układ jest odizolowany od otoczenia, gdyż wtedy jego całkowita energia nie zmienia się w czasie. Matematycznym wyrazem tego jest, że operator Hamiltona nie zależy jawnie od czasu, lecz jest wyrażony tylko przez pochodne względem r Wtedy wektor stanu przyjmuje postać iloczynu, zawierającego czynnik zależny od czasu i czynnik zależny tylko od położenia 
\begin{math}
{ \Psi (r,t)=\exp \left(-{\frac {i}{\hbar }}Et\right)\psi (r),}
\end{math}

Wstawiając powyższą postać funkcji falowej do równania ogólnego, otrzymuje się równanie Schrödingera niezależne od czasu 

$$ {\hat  {H}}\psi (r)=E\psi (r). $$

\section{Teoretyczne obliczenie dyskretnych energii}
Konieczność teoretycznego wprowadzenia kwantyzacji układów fizycznych na początku XX wieku wynikła z nowo odkrytych faktów doświadczalnych. Stwierdzono bowiem, że niektóre układy fizyczne nie przyjmują dowolnych wartości energii, a jedynie wartości dyskretne. Odkryto na przykład, że pojedyncze atomy dają dyskretne widmo promieniowania, ciała stałe emitują promieniowanie termiczne o tzw. rozkładzie ciała doskonale czarnego, które dało się wyjaśnić, jedynie przyjmując emisję promieniowanie w postaci dyskretnych porcji – fotonów; jony w ciałach stałych mają dyskretne energie, co ma wpływ na charakterystyczną wartość ich ciepła właściwego.

Schrödinger pokazał, że można teoretycznie obliczyć dyskretne wartości energii w przypadku stanów związanych układu. Mianowicie, jeżeli operator Hamiltona zapisze się w bazie jego stanów własnych, to niezależne od czasu równanie Schrödingera przybiera postać macierzową 
${ H_{\mu }^{\nu }c_{\nu }=Ec_{\mu },}$
gdzie ${ H_{\mu }^{\nu }=\langle \mu |H|\nu \rangle .}$ Równanie powyższe jest układem równań liniowych n-tego rzędu. Przekształcając je do postaci:
$${ (H_{\mu }^{\nu }-E\delta _{\mu }^{\nu })c_{\nu }=0,}$$

widzimy, że układ ten ma rozwiązania niezerowe jedynie wtedy, gdy wyznacznik główny układu jest równy zero, tzn. 
$${ \det(\mathbf {H} -E\mathbf {I} )=0.}$$

\section{Przykłady hamiltonianu}
Hamiltonian składa się z sumy operatorów energii kinetycznych cząstek układu oraz sumy energii potencjalnych, związanych z oddziaływaniami cząstek układu ze sobą i z polem zewnętrznym (np. polem elektromagnetycznym).
Cząstka o masie m w polu potencjalnym.
W przypadku cząstki nierelatywistycznej (tj. poruszającej się z prędkością znacznie mniejszą od prędkości światła, v c), oraz pozbawionej ładunku elektrycznego i spinu, operator energii kinetycznej ma postać: 
$${ {\hat {T}}={\frac {{\hat {\mathbf {p} }}^{2}}{2m}}=-{\frac {\hbar ^{2}}{2m}}\nabla ^{2},}$$
gdzie:
$${ {\hat {\mathbf {p} }}=-i\hbar \nabla =-i\hbar \left({\frac {\partial }{\partial x}},{\frac {\partial }{\partial y}},{\frac {\partial }{\partial z}}\right)}$$
Działanie operatora energii potencjalnej V który w ogólności zależy od położenia r w przestrzeni oraz chwili czasu t . 
\end{document}