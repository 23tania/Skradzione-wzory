\documentclass{article}
\usepackage[utf8]{inputenc}
\usepackage[MeX,plmath]{polski} 
\usepackage{amsmath}
\usepackage{amsfonts}

\begin{document}
\title{Tożsamość Brahmagupty}
\maketitle

\section*{Wprowadzenie}
Tożsamość Brahmagupty, zwana również tożsamością Fibonacciego, stwierdza, że iloczyn dwóch sum dwóch kwadratów jest również sumą dwóch kwadratów. Oznacza to, że zbiór wszystkich sum dwóch kwadratów jest zamknięty ze względu na mnożenie: 
\begin{displaymath}
\left(a^{2}+b^{2}\right)\left(c^{2}+d^{2}\right)=\left(ac-bd\right)^{2}+\left(ad+bc\right)^{2} =\left(ac+bd\right)^{2}+\left(ad-bc\right)^{2}
\end{displaymath}

Na przykład:
\[ (1^{2}+4^{2})(2^{2}+7^{2})=26^{2}+15^{2}=30^{2}+1^{2} \]

Tożsamość jest specjalnym przypadkiem (n = 1) tożsamości Lagrange’a, i po raz pierwszy pojawia się w dziełach Diofantosa. Brahmagupta udowodnił ogólniejszą równość, w równoważnej formie: 
\begin{multline*}
 \left(a^{2}+nb^{2}\right)\left(c^{2}+nd^{2}\right)=\left(ac-nbd\right)^{2}+n\left(ad+bc\right)^{2}\\
 =\left(ac+nbd\right)^{2}+n\left(ad-bc\right)^{2}
\end{multline*} 

Obie równości można udowodnić poprzez rozwinięcie obu stron równania. 

Równość zachodzi dla liczb całkowitych, wymiernych, rzeczywistych i ogólnie, dla dowolnego pierścienia przemiennego. 

W przypadku całkowitym, tożsamość znajduje zastosowanie w teorii liczb; jeśli użyje się jej wraz z twierdzeniem Fermata o sumie dwóch kwadratów, można udowodnić, że iloczyn kwadratu i dowolnie wielu liczb pierwszych postaci 4n + 1 jest sumą dwóch kwadratów. 

\section*{Historia}
Tożsamość po raz pierwszy pojawia się w Arytmetyce Diofantosa (III, 19). Została ponownie odkryta przez Brahmaguptę (598–668), indyjskiego matematyka i astronoma, który uogólnił ją i używał do badań równań błędnie nazywanych równaniami Pella. Jego Brahmasphutasiddhanta została przetłumaczona z Sanskrytu na arabski przez by Mohammada al-Fazariego, a potem na łacinę w 1126. Tożsamość pojawia się później w książce Fibonacciego Liber quadratorum z 1225 roku. 

\section*{Związek z liczbami zespolonymi}
Jeśli a, b, c i d są liczbami rzeczywistymi, tożsamość równoważna jest multiplikatywności modułu w ciele liczb zespolonych:
$$ |a+bi|\ |c+di|=|(a+bi)(c+di)| $$
Ponieważ
$$  |a+bi|\ |c+di|=|(ac-bd)+i(ad+bc)| $$
więc po podniesieniu obu stron do kwadratu,
$$ |a+bi|^{2}|c+di|^{2}=|(ac-bd)+i(ad+bc)|^{2} $$
i po zastosowaniu definicji modułu, otrzymamy: 
$$ (a^{2}+b^{2})(c^{2}+d^{2})=(ac-bd)^{2}+(ad+bc)^{2} $$

\section*{Zastosowanie do rozwiązywania równań Pella}
Brahmagupta zastosował odkrytą tożsamość do rozwiązania konkretnego równania Pella: $ x^{2}-Ny^{2}=1 $. Używając tożsamości w ogólniejszej postaci:
\[ (x_{1}^{2}-Ny_{1}^{2})(x_{2}^{2}-Ny_{2}^{2})=(x_{1}x_{2}+Ny_{1}y_{2})^{2}-N(x_{1}y_{2}+x_{2}y_{1})^{2} \]
Brahmagupta był w stanie otrzymać nową trójkę:
\[ (x_{1}x_{2}+Ny_{1}y_{2}\,,\,x_{1}y_{2}+x_{2}y_{1}\,,\,k_{1}k_{2}) \]

Metoda ta nie tylko pozwoliła na otrzymanie nieskończenie wielu rozwiązań równania przy użyciu tylko jednego rozwiązania, ale także na uzyskanie całkowitych, lub „prawie całkowitych” wyników, poprzez podzielenie otrzymanej trójki liczb. Ogólna metoda rozwiązywania równań Pella (tzw. metoda ćakrawala) została znaleziona przez Bhaskarę II w 1150 i bazowała ona na tożsamości Brahmagupty.

\end{document}