\documentclass{article}
\usepackage[utf8]{inputenc}
\usepackage[MeX,plmath]{polski} 
\usepackage{amsmath}
\usepackage{amsfonts}

\begin{document}
\title{Funkcja wykładnicza}
\maketitle

\section*{Wprowadzenie}
Funkcja wykładnicza – funkcja postaci:
$$  f(x)=a^{x}, {\text{ gdzie }} \: a>0 $$
Niektórzy autorzy wymagają, aby podstawa a funkcji wykładniczej była różna od 1, ponieważ dla  a = 1 funkcja  ax jest funkcją stałą. 

\section*{Własności}
\begin{itemize}
\item $ a^{x+y}=a^{x}\cdot a^{y} $
\item \begin{math} a^{x-y}={\frac {a^{x}}{a^{y}}} \end{math}
\item Pochodna funkcji wykładniczej to:
\begin{displaymath}
(a^{x})'=\lim _{\Delta x\to 0}{\frac {a^{x+\Delta x}-a^{x}}{\Delta x}}=\lim _{\Delta x\to 0}a^{x}{\frac {a^{\Delta x}-1}{\Delta x}}=a^{x}\lim _{\Delta x\to 0}{\frac {a^{\Delta x}-1}{\Delta x}}=a^{x}\ln a
\end{displaymath}
\end{itemize}

\section*{Funkcja eksponencjalna}
Szczególnym przypadkiem funkcji wykładniczej jest funkcja eksponencjalna, czyli funkcja wykładnicza o podstawie równej e (czyli podstawie logarytmu naturalnego). Innym oznaczeniem takiej funkcji jest exp(x) (nazywane skrótowo eksponentą). 

Cechą funkcji $ f(x)=e^{x} $ jest to, że jej pochodna jest równa jej samej. Zastosowanie metody łamanych Eulera do rozwiązywania równania różniczkowego: 
\begin{equation*}
{\dot  {x}}=x
\end{equation*}
daje wzór na funkcję eksponencjalną: 
$$ \exp(x)=\lim _{n\to \infty }\left(1+{\tfrac {x}{n}}\right)^{n} $$

Eksponens jako funkcję analityczną na mocy twierdzenia Taylora można rozwinąć w szereg potęgowy: \( \sum _{n=0}^{\infty }{\frac {x^{n}}{n!}} \)

\subsection*{Płaszczyzna zespolona}
Funkcję eksponencjalną łatwo uogólnić na ciało liczb zespolonych. Jedną z metod jest wykorzystanie rozwinięcia funkcji w szereg Taylora i podstawienie zespolonego argumentu w miejsce rzeczywistego:
\[ e^{z}=\sum _{n=0}^{\infty }{\frac {z^{n}}{n!}} \]

Można ją zapisać jako: 
$$ e^{a+bi}=e^{a}(\cos b+i\sin b) $$
gdzie a i b to odpowiednio współczynniki części rzeczywistej i urojonej danej liczby zespolonej. 

\end{document}