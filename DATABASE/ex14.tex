\documentclass{article}
\usepackage[utf8]{inputenc}

\begin{document}
\title{Liczby zespolone - Postać wykładnicza}
\maketitle

\section{Liczby zespolone - Postać wykładnicza}
Rzeczywiste funkcje sin , cos , oraz exp zmiennej rzeczywistej można rozwinąć na szeregi Maclaurina: 
$${ \sin x=\sum _{n=0}^{\infty }{\frac {(-1)^{n}x^{2n+1}}{(2n+1)!}},\ \cos x=\sum _{n=0}^{\infty }{\frac {(-1)^{n}x^{2n}}{(2n)!}},\ e^{x}=\sum _{n=0}^{\infty }{\frac {x^{n}}{n!}}}$$
które są zbieżne dla każdego x ∈ R . Ponieważ w tych wzorach występują jedynie działania dodawania, mnożenia, dzielenia i podnoszenia do potęgi o wykładniku naturalnym, które są dobrze zdefiniowane dla liczb zespolonych, to wzory te mogą posłużyć jako definicje zespolonych funkcji zmiennej zespolonej. Mianowicie definiuje się funkcje: 
$${ \cos :\mathbb {C} \to \mathbb {C} ,\ \cos z:=\sum _{n=0}^{\infty }{\frac {(-1)^{n}z^{2n}}{(2n)!}}}$$
$${ \exp :\mathbb {C} \to \mathbb {C} ,\ \exp z:=\sum _{n=0}^{\infty }{\frac {z^{n}}{n!}}}$$
Definicje te są poprawne, ponieważ szeregi występujące po prawej stronie są zbieżne dla każdego z ∈ C , gdyż kryteria zbieżności szeregów takie jak kryterium d’Alemberta i kryterium Cauchy’ego pozostają prawdziwe dla liczb zespolonych[7].

Korzystając z pojęcia iloczynu Cauchy’ego szeregów, można udowodnić, że: 
$${ \exp(z+w)=\exp(z)\exp(w)}$$
Z definicji oraz własności szeregów wynikają następujące wzory: 
$${ \sin(-z)=-\sin z,\ \cos(-z)=\cos z,\ e^{iz}=\cos z+i\sin z}$$
W szczególności: ${ e^{ix}=\cos x+i\sin x,}$
Zatem każda liczba zespolona różna od zera ma następujące przedstawienie: 
$${ z=|z|(\cos \varphi +i\sin \varphi )=|z|e^{i\varphi },}$$
Pierwiastki zespolone w postaci wykładniczej wyrażają się wzorami: 
$z_{k}={\sqrt[ {n}]{|z|}}\ e^{{i{\tfrac  {\varphi +2k\pi }{n}}}}$
Korzystając z parzystości cosinusa i nieparzystości sinusa, można też wyprowadzić następujące wzory na funkcje trygonometryczne: 
$${ \sin z={\frac {e^{iz}-e^{-iz}}{2i}}}$$
$${ \cos z={\frac {e^{iz}+e^{-iz}}{2}}}$$
\end{document}