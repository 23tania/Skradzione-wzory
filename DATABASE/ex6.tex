\documentclass{article}
\usepackage[utf8]{inputenc}

\begin{document}
\title{Metoda Cayleya}
\maketitle

\section{Metoda Cayleya}
Metoda Cayleya – w mechanice kwantowej popularna metoda numerycznego rozwiązywania równania Schrödingera zależnego od czasu polegająca na przybliżeniu propagatora w czasie poprzez łatwiejszy do obliczenia niż dokładny operator unitarny, tzn. tak aby operator przybliżony też nie zmieniał normy funkcji falowej.

Dla równania Schrödingera zależnego od czasu ( h = 1 ) :

    $$ { {\hat {H}}{\big |}\psi (t)\rangle =i{\tfrac {\partial }{\partial t}}{\big |}\psi (t)\rangle ,}$$

funkcja falowa dla małych czasów będzie dana przez:

    $$ { |\psi (t+dt)\rangle =e^{-i{\hat {H}}(t)dt}|\psi (t)\rangle .}$$

Zauważamy

    $$ { U_{1}=e^{-i{\hat {H}}(t)dt}=1-i{\hat {H}}(t)dt+{\hat {H}}(t)^{2}dt^{2}/2+i{\hat {H}}(t)^{3}dt^{3}/6+...}$$

oraz

    $$ { U_{2}={\frac {1-i{\hat {H}}(t)dt/2}{1+i{\hat {H}}(t)dt/2}}=1-i{\hat {H}}(t)dt+{\hat {H}}(t)^{2}dt^{2}/2+i{\hat {H}}(t)^{3}dt^{3}/4+...}$$

Jak widać powyższe operatory są równe do drugiego rzędu w d t dt oraz operator U 2  jest z konstrukcji unitarny (jest ułamkiem) tak samo jak operator dokładny, tzn.

    $$ { U_{2}^{*}=U_{2}^{-1}.}$$

Używamy więc

    $$⟩ |\psi(t+dt)\rangle=U_2|\psi(t)\rangle $$

Powyższy krok w przybliżeniu Cranka-Nicolson jest wtedy układem równań liniowych na funkcje $\psi(t+dt)$  w chwili czasu ${ t+dt,}$ ,  tzn.

    $$ { [{1+i{\hat {H}}(t)dt/2}]|\psi (t+dt)\rangle =[{1-i{\hat {H}}(t)dt/2}]|\psi (t)\rangle }$$

i jest powtarzany wielokrotnie numerycznie. 
\end{document}