\documentclass{article}
\usepackage[utf8]{inputenc}

\begin{document}
\title{Równanie Schrödingera}
\maketitle

\section{Równanie Schrödingera}
Równanie Diraca – jedno z fundamentalnych równań w relatywistycznej mechanice kwantowej, sformułowane przez angielskiego fizyka Paula Diraca w 1928 roku, słuszne dla cząstek o dowolnie wielkich energiach (tzw. cząstek relatywistycznych) o spinie 1/2 (fermiony, np. elektrony, kwarki), swobodnych i oddziałujących z polem elektromagnetycznym. Istnienie spinu wynika z samego żądania relatywistycznej niezmienniczości równania ruchu cząstek. Odpowiada równaniu Pauliego, które także zawiera spin cząstek, ale wprowadza go w sposób fenomenologiczny, niejako sztuczny, a jedynie dlatego, by otrzymać zgodność z doświadczeniem Sterna-Gerlacha (rozszerzając formalizm nierelatywistycznego równania Schrödingera).

Równanie Diraca jest równaniem macierzowym – de facto stanowi ono układ 4 równań ze względu na fakt, iż symbole gamma (lub alfa, beta), występujące w tym równaniu, są macierzami 4 × 4.

Równania Diraca zapisuje się w postaci jawnie relatywistycznie niezmienniczej lub w tzw. obrazie Schrödingera. Ta ostatnia postać została najpierw wyprowadzona przez Diraca i jest stosowana ze względu na wygodę do wykonywania obliczeń, gdyż odróżnia współrzędne przestrzenne od współrzędnej czasowej.

Równanie Diraca zostało potwierdzone w odniesieniu do struktury subtelnej widma atomu wodoru, wykazując znakomitą zgodność z pomiarami. Przewiduje istnienie antycząstek. Niektóre jednak efekty, takie jak kreacja i anihilacja cząstek czy przesunięcie Lamba tłumaczy dopiero elektrodynamika kwantowa. 


\section{Macierze gamma γ }
Macierze gamma γ μ to macierze zespolone 4 × 4 spełniające 16 reguł antykomutacyjnych w postaci.

$${ \left\{\gamma ^{\mu },\gamma ^{\nu }\right\}=2g^{\mu \nu }I,}$$

$${ \left\{A,B\right\}=AB+BA}$$

Równanie cząstki swobodnej

W zapisie jawnie relatywistycznie niezmienniczym równanie Diraca dla cząstki swobodnej ma postać
$${ (i\hbar \,\gamma ^{\mu }\partial _{\mu }-mc)\Psi (x^{\nu })=0,}$$
gdzie:
$${ x^{\nu }=(x^{0}=ct,x^{1},x^{2},x^{3})}$$
$${ \partial _{\mu }={\frac {\partial }{\partial x^{\mu }}}}$$
$${ \partial _{\mu }=_{,\mu }={\frac {\partial }{\partial x^{\mu }}}=\left({\frac {1}{c}}{\frac {\partial }{\partial t}},{\vec {\nabla }}\right)=\left({\frac {\partial _{t}}{c}},{\vec {\nabla }}\right),}$$

\section{Bispinor}
Definiuje się bispinor Ψ  hermitowsko sprzężony do bispinora Ψ  – przedstawia on wektor w postaci wiesza, którego elementami są sprzężenia zespolone składowych bispinora (przy czym † oznacza sprzężenie hermitowskie) 
$${ \Psi ^{\dagger }=[\,\psi _{1}^{*},\psi _{2}^{*},\psi _{3}^{*},\psi _{4}^{*}\,].}$$

\section{Gęstość prawdopodobieństwa w teorii Diraca}
Gęstość prawdopodobieństwa definiuje się analogicznie jak w teorii Schrödingera 
$${ \rho =\Psi ^{\dagger }\Psi .}$$

W definicji gęstości prawdopodobieństwa dla równania Diraca istotna jest kolejność czynników: Ψ † musi być przed Ψ , gdyż występuje tu mnożenie wektorów w postaci wiersza i kolumny, i tylko dla takiej kolejności mnożenie da w wyniku skalar. (W analogicznym wyrażeniu na gęstość prawdopodobieństwa dla równania Schrödingera funkcja falowa jest skalarem, stąd kolejność mnożenia nie ma znaczenia).

Wykonując obliczenia otrzymamy 
$$y_1=\varepsilon_1\cdot v_0+\varepsilon_1^2\cdot u_0$$
$$(v_0)^3\cdot (u_*)^3=\frac{-q+\sqrt{q^2+4p^3/27}}{2}\cdot \frac{-q-\sqrt{q^2+4p^3/27}}{2}=\frac{q^2-q^2-4p^3/27}{4}=\frac{-p^3}{27}.$$
\end{document}