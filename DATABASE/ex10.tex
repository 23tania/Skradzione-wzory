\documentclass{article}
\usepackage[utf8]{inputenc}

\begin{document}
\title{Równanie sześcienne}
\maketitle

\section{Równanie sześcienne}
Równanie sześcienne lub trzeciego stopnia – równanie algebraiczne postaci a $ax^3+bx^2+cx+d=0,$ Każde równanie sześcienne o współczynnikach rzeczywistych ma przynajmniej jeden pierwiastek rzeczywisty. 
\section{Rozwiązywanie równań kanonicznych}
Zwróćmy uwagę, że jeśli znajdziemy jeden pierwiastek y 0 równania 
$${ y^{3}+py+q=0,}$$
to na mocy tzw. twierdzenia Bézouta możemy podzielić wielomian y 3 + p y + q  redukując nasze równanie do równania kwadratowego. Rozwiązując to równanie możemy znaleźć pozostałe rozwiązania równania (2). Poniżej najpierw przedstawimy metodę znajdowania jednego pierwiastka naszego równania, a później bardziej szczegółowo opiszemy sposób na znajdowanie wszystkich rozwiązań tego równania. 
Jak znaleźć jeden pierwiastek
Rozważamy równanie 
$${ y^{3}=v^{3}+3uv^{2}+3u^{2}v+u^{3}=3uv(u+v)+u^{3}+v^{3}=3uvy+u^{3}+v^{3}.}$$
Po dalszym uporządkowaniu informacji ze wzoru (3) otrzymujemy równanie 
$${ y^{3}-3uvy-(u^{3}+v^{3})=0.}$$
Zauważamy, że jeśli 
$${ \left({\frac {-p}{3v}}\right)^{3}+v^{3}+q=0.}$$
Stąd
$${ v^{3}-{\frac {p^{3}}{27v^{3}}}+q=0.}$$
Po pomnożeniu przez v 3 otrzymamy:
$${ (v^{3})^{2}+q(v^{3})-{\frac {p^{3}}{27}}=0.}$$
Podstawiając za v 3 zmienną pomocniczą  z, otrzymujemy równanie kwadratowe:
$${ z^{2}+qz-{\frac {p^{3}}{27}}=0.}$$
Równanie (6) ma pierwiastek (możliwe że zespolony): 
$$z_0=\frac{-q+\sqrt{q^2+4p^3/27}}{2}.$$
$$\varepsilon_1=\frac{-1 + i\sqrt 3}{2},$$
\end{document}