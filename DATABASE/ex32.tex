\documentclass{article}
\usepackage[utf8]{inputenc}
\usepackage[MeX,plmath]{polski} 
\usepackage{amsmath}
\usepackage{amsfonts}

\begin{document}
\title{Twierdzenie Lagrange’a (rachunek różniczkowy)}
\maketitle

\section*{Wprowadzenie}
Twierdzenie Lagrange’a – jedno z kilku twierdzeń o wartości średniej w rachunku różniczkowym.

Nazwa twierdzenia pochodzi od nazwiska Josepha Louisa Lagrange’a.

\section*{Twierdzenie Lagrange'a}
Jeśli dana funkcja f : [ a , b ] → R jest ciągła w przedziale [a,b], różniczkowalna w przedziale (a, b),

to istnieje taki punkt c należący do (a,b)  że:

\begin{equation}
	{\frac {f(b)-f(a)}{b-a}}=f'(c)
\end{equation}

Twierdzenie nie zachodzi w przypadku wielowymiarowym. 

\section*{Wartość średnia}

Twierdzenie Lagrange’a zapisane w postaci

\begin{displaymath}
	f(b)-f(a)=f'(c)(b-a)
\end{displaymath}

mówi, że przyrost wartości funkcji dla argumentów b i a wyraża się przez przyrost wartości zmiennej (argumentów) i pochodną funkcji w pewnym punkcie pośrednim między a i b – stąd właśnie nazwa twierdzenia. 

\section*{Dowód}
Kładziemy:

$$ K={\frac {f(b)-f(a)}{b-a}} $$
$$ g(x)=f(x)-K(x-a) $$

Mamy wtedy: \(g(a)=f(a)-K(a-a)=f(a)\) 
oraz: \[g(b)=f(b)-K(b-a)=f(b)-f(b)+f(a)=f(a)\]

A więc g(a)=g(b), czyli funkcja g(x) spełnia założenia twierdzenia Rolle’a, a zatem istnieje punkt c należący do (a,b) taki, że $g' (c)=0$ , z drugiej strony mamy \(g'(x)=f'(x)-K\) i stąd otrzymujemy \(0=g'(c)=f'(c)-K\). Dlatego też $f'(c)=K={\frac {f(b)-f(a)}{b-a}}$. 

\section*{Uogólnienie}
Dla funkcji o wartościach w dowolnych przestrzeniach wektorowych (a nawet w R do n dla n większe od 1 teza twierdzenia nie jest spełniona. Na przykład linia śrubowa (traktowana jako wykres funkcji R → R do 2 podczas jednego obrotu nie ma w żadnym momencie pochodnej zerowej, a powraca do swojej wartości (na osiach x,y). 

Dowód polega na stwierdzeniu, że dla każdego e większe od 0 i każdego ograniczenia górnego normy pochodnej na przedziale (a,b) kresem górnym zbioru końców przedziału dla których teza przy a+e w miejscu a i ograniczeniu górnym zamiast supremum jest spełniona, jest b. Po zastąpieniu ograniczenia kresem, nierówność pozostanie spełniona. Przejście graniczne z e do zera daje, dzięki ciągłości funkcji f tezę. 

\end{document}