\documentclass{article}
\usepackage[utf8]{inputenc}
\usepackage[MeX,plmath]{polski} 
\usepackage{amsmath}
\usepackage{amsfonts}

\begin{document}
\title{Wzór Eulera}
\maketitle

\section*{Wprowadzenie}
Wzór Eulera – wzór analizy zespolonej wiążący funkcje trygonometryczne z zespoloną funkcją wykładniczą, określany nazwiskiem Leonharda Eulera.  

\section*{Wzór}

Niech x należy do liczb rzeczywistych zaś i jest jednostką urojoną, wtedy wzór Eulera ma postać

\[e^{ix}=\cos x+i\sin x\]

\section*{Historia}
Wzór Eulera został dowiedziony po raz pierwszy przez Rogera Cotesa w 1714 w postaci

\begin{equation}
	\ln(\cos x+i\sin x)=ix
\end{equation}

Euler był pierwszym, który opublikował go w obecnie stosowanej formie w 1748, opierając swój dowód na równości szeregów po obu stronach tożsamości. Żaden z nich nie widział interpretacji geometrycznej tego wzoru: utożsamienie liczb zespolonych z płaszczyzną zespoloną powstało około 50 lat później (wynik Caspara Wessela).

\section*{Dowód}
Rozwinięte w szereg potęgowy funkcje mają postać:

\begin{displaymath}
	e^{x}=1+x+{\frac {x^{2}}{2!}}+{\frac {x^{3}}{3!}}+{\frac {x^{4}}{4!}}+\dots =\sum _{n=0}^{\infty }{\frac {x^{n}}{n!}}
\end{displaymath}

\begin{equation*}
	\sin x=x-{\frac {x^{3}}{3!}}+{\frac {x^{5}}{5!}}-\dots =\sum _{n=0}^{\infty }{\frac {(-1)^{n}x^{2n+1}}{(2n+1)!}}
\end{equation*}

\[\cos x=1-{\frac {x^{2}}{2!}}+{\frac {x^{4}}{4!}}-\dots =\sum _{n=0}^{\infty }{\frac {(-1)^{n}x^{2n}}{(2n)!}}\]

Powyższe wzory służą jako definicje zespolonych funkcji exp, sin i cos, tzn. definiuje się funkcje: 


\begin{equation}
	\exp :\mathbb {C} \to \mathbb {C} ,\ \exp z:=\sum _{n=0}^{\infty }{\frac {z^{n}}{n!}}
\end{equation}


\begin{displaymath}
	\sin :\ \mathbb {C} \to \mathbb {C} ,\ \sin z:=\sum _{n=0}^{\infty }{\frac {(-1)^{n}z^{2n+1}}{(2n+1)!}}
\end{displaymath}


\begin{equation*}
	\cos :\mathbb {C} \to \mathbb {C} ,\ \cos z:=\sum _{n=0}^{\infty }{\frac {(-1)^{n}z^{2n}}{(2n)!}}
\end{equation*}

Definicje te są poprawne, ponieważ szeregi występujące po prawej stronie są zbieżne dla każdego z należącego do liczb zespolonych, gdyż kryteria zbieżności szeregów takie jak kryterium d’Alemberta i kryterium Cauchy’ego pozostają prawdziwe dla liczb zespolonych.

W szczególności mamy: 

\begin{equation}
	e^{iz}=1+iz+{\frac {(iz)^{2}}{2!}}+{\frac {(iz)^{3}}{3!}}+\dots =\left(1-{\frac {z^{2}}{2!}}+{\frac {z^{4}}{4!}}-\dots \right)+i\left(z-{\frac {z^{3}}{3!}}+{\frac {z^{5}}{5!}}-\dots \right)=\cos z+i\sin z
\end{equation}

\section*{Tożsamość Eulera}
W szczególności, podstawiając x = pi, otrzymuje się równość:

\begin{displaymath}
	e^{\pi i}+1=0
\end{displaymath}

nazywaną też tożsamością Eulera (czasami wzorem Eulera).

Nie istnieją żadne znane dokumenty potwierdzające autorstwo Eulera; co więcej, była ona zapewne znana matematykom żyjącym przed nim. 

„Najpiękniejszy wzór”

Tożsamość Eulera nazywana jest często najpiękniejszym wzorem matematycznym. Wykorzystane są w niej trzy działania arytmetyczne: dodawanie, mnożenie i potęgowanie. Tożsamość łączy pięć fundamentalnych stałych matematycznych:

- liczbę 0, \\
- liczbę 1, \\
- liczbę pi, \\
- liczbę e, \\
- liczbę i, jednostkę urojoną liczb zespolonych. \\

Dodatkowo każde z powyższych działań oraz każda ze stałych użyte są dokładnie raz, co więcej: wzór ten jest przedstawiony w zwyczajowej formie równania, którego prawa strona jest zerem. 

\end{document}