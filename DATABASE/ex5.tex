\documentclass{article}
\usepackage[utf8]{inputenc}

\begin{document}
\title{Metoda Hartree-Focka}
\maketitle

\section{Metoda Hartree-Focka}
Metoda Hartree-Focka (poprawnie: Hartree'ego-Foka[1], też metoda pola samouzgodnionego, metoda HF) – jedna z metod przybliżonego rozwiązywania problemów wielu ciał w mechanice kwantowej wielu cząstek. Metoda ta została opracowana przez D.R. Hartreego, zanim dostępne były komputery, a następnie poprawiona tak, aby uwzględniać zakaz Pauliego, przez W. Foka[2].

Metoda Hartree-Focka jest powszechnie stosowana w chemii kwantowej, fizyce jądrowej, fizyce atomu i fizyce materii skondensowanej, gdzie pozwala na przybliżone rozwiązanie równania Schrödingera dla układu wielu cząstek. Jest to podstawowa metoda obliczeniowa ab initio. Oblicza się w niej energię i funkcję falową stanu podstawowego układu wielu cząstek (np. energię elektronową wieloelektronowego atomu lub cząsteczki) w oparciu o model cząstek niezależnych (w przypadku elektronów zwany przybliżeniem jednoelektronowym).

Metoda Hartree-Focka oparta jest na zasadzie wariacyjnej głoszącej, iż energia stanu obliczona jako wartość oczekiwana z dowolnej funkcji falowej jest zawsze większa bądź równa energii będącej dokładnym rozwiązaniem równania Schrödingera. Zakłada się w niej, że funkcja falowa jest, w przypadku układu N identycznych fermionów (np. elektronów), wyznacznikiem macierzy zbudowanej z funkcji zależnych od współrzędnych poszczególnych cząstek ϕ i ( τ j ) (zwanych spinorbitalami). Wyznacznik taki nosi nazwę wyznacznika Slatera. 

\section{Równania Hartree-Focka i operator Focka}
Znalezienie orbitali Hartree-Focka sprowadza się do rozwiązania układu równań Hartree-Focka, o postaci podobnej do niezależnego od czasu równania Schrödingera
$$ { {\hat {F}}\phi _{i}(\tau _{i})=\epsilon _{i}\phi _{i}(\tau _{i})} $$
Operator F zwany jest operatorem Focka, i ma postać:
$$ { {\hat {F}}=-{\frac {\hbar ^{2}}{2m}}\nabla _{\mathbf {r} _{i}}^{2}+{\hat {V}}_{eN}(\mathbf {r} _{i})+e^{2}\sum \limits _{j}{\hat {J}}_{j}-{\hat {K}}_{j}\equiv {\hat {h}}_{i}+e^{2}\sum \limits _{j}{\hat {J}}_{j}-{\hat {K}}_{j}} $$
gdzie operator $ { -{\frac {\hbar ^{2}}{2m}}\nabla _{\mathbf {r} _{i}}^{2}}$   jest operatorem energii kinetycznej elektronu.
Operator J  jest operatorem oddziaływania elektrostatycznego elektronu i z elektronem j i jego działanie na ϕ i sprowadza się do pomnożenia przez całkę:
$$ { {\hat {J}}_{j}\phi _{i}(\tau _{1})=\int \phi _{j}^{*}(\tau _{2}){\frac {1}{\vert \mathbf {r} _{2}-\mathbf {r} _{1}\vert }}\phi _{j}(\tau _{2})d\tau _{2}\cdot \phi _{i}(\tau _{1})} $$
Operator K ,  zwany operatorem wymiennym, nie ma odpowiednika klasycznego.
$$ { {\hat {K}}_{j}\phi _{i}(\tau _{1})=\int \phi _{j}^{*}(\tau _{2}){\frac {1}{\vert \mathbf {r} _{2}-\mathbf {r} _{1}\vert }}\phi _{i}(\tau _{2})d\tau _{2}\cdot \phi _{j}(\tau _{1})} $$

\section{Energia elektronowa w metodzie Hartree-Focka}
Całkowita energia elektronowa w metodzie Hartree-Focka wynosi 
$${ E=\sum _{i=1}^{n}I_{i}+\sum _{i>j=1}^{n}\left(J_{ij}-K_{ij}\right)}$$
$${ J_{ij}=\int \phi _{i}^{*}(i)\phi _{j}^{*}(j){\frac {e^{2}}{r_{ij}}}\phi _{i}(i)\phi _{j}(j)d\tau _{i}\tau _{j}}$$
$${ K_{ij}=\int \phi _{i}^{*}(i)\phi _{j}^{*}(j){\frac {e^{2}}{r_{ij}}}\phi _{i}(j)\phi _{j}(i)d\tau _{i}\tau _{j}}$$
Całka wymienna K i j ,  obniżająca energię, jest różna od zera tylko dla elektronów o spinach skierowanych równolegle. W konsekwencji stan trypletowy ma zawsze mniejszą energię niż stan singletowy o tej samej konfiguracji elektronowej (reguła Hunda). 
$${ E=\sum _{i=1}^{n/2}I_{i}^{'}+\sum _{i,j=1}^{n/2}\left(2J_{ij}^{'}-K_{ij}^{'}\right)}$$
Z energiami orbitalnymi orbitali zajętych ϵ i całkowita energia elektronowa układu zamkniętopowłokowego związana jest następującą zależnością: 
$${ E=2\sum _{i=1}^{n/2}\epsilon _{i}+\sum _{i,j=1}^{n/2}\left(2J_{ij}^{'}-K_{ij}^{'}\right)}$$
\end{document}