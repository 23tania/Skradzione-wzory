\documentclass{article}
\usepackage[utf8]{inputenc}
\usepackage[MeX,plmath]{polski} 
\usepackage{amsmath}
\usepackage{amsfonts}

\begin{document}
\title{Transformacja Laplace'a}
\maketitle
\section*{Definicja podstawowa}
Jednostronną transformatą Laplace’a funkcji $ \mathbb{R} \ni t \mapsto f(t) \in \mathbb{R} $ nazywamy następującą funkcję \( \mathbb{C} \ni s \mapsto F(s) \in \mathbb{C} \)
$$ F(s) = \left\{\mathcal{L} f\right\}(s) = \int\limits_0^\infty e^{-st} f(t)\, dt. $$
często zapisywaną, zwłaszcza w środowisku inżynierskim, w następującej formie: 
\[ F(s) = \mathcal{L} \left\{f(t)\right\} = \int\limits_0^\infty e^{-st} f(t)\, dt. \]

Należy zwrócić uwagę na rozróżnienie pomiędzy pojęciem transformaty a transformacji Laplace’a. Zgodnie z powyższą definicją transformacja Laplace’a jest przekształceniem zbioru funkcji, dla których całka Laplace’a jest zbieżna w zbiór funkcji zespolonych zmiennej zespolonej. Natomiast transformata Laplace’a jest jedynie obrazem pewnej funkcji f(t) przez transformację Laplace’a.

Matematykiem, który zdefiniował transformację Laplace’a i od którego nazwiska wzięła ona nazwę był Pierre Simon de Laplace. 

\section*{Własności}
\begin{itemize}
    \item liniowość
    
    \begin{displaymath}
    \mathcal{L}\left\{a f(t) + b g(t) \right\}= a \mathcal{L}\left\{ f(t) \right\} + b \mathcal{L}\left\{ g(t) \right\}
    \end{displaymath}

    
    \item pochodna transformaty
    
    \begin{equation*}
         F^{(n)}(s)=(-1)^{n}{\mathcal {L}}\{t^{n}f(t)\}
    \end{equation*}
    
    \item transformata pochodnej
    
    \begin{gather*}
    {\mathcal {L}}\{f'\}=s{\mathcal {L}}\{f\}-f(0^{+})
    \end{gather*}
    $$ \mathcal{L}\{f''\} = s^2 \mathcal{L}\{f\} - s f(0^+) - f'(0^+) $$
    
    \item transformata całki
    
    \begin{eqnarray*}
    \mathcal{L}\left\{ \frac{f(t)}{t} \right\} = \int\limits_s^\infty F(\sigma)\, d\sigma
    \end{eqnarray*}
    
    \item splot jednostronny
    
    \begin{multline*}
        {\mathcal {L}}\left\{\int \limits _{0}^{t}f(u)\cdot g(t-u)\,du\right\}=\\{\mathcal {L}}\{f*g\}={\mathcal {L}}\{f\}{\mathcal {L}}\{g\}
    \end{multline*}
\end{itemize}

\end{document}