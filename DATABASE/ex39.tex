\documentclass{article}
\usepackage[utf8]{inputenc}
\usepackage[MeX,plmath]{polski} 
\usepackage{amsmath}
\usepackage{amsfonts}

\begin{document}
\title{Twierdzenie Taubera}
\maketitle

\section*{Wprowadzenie}
Twierdzenie Taubera – twierdzenie analizy zespolonej pozwalające odwrócić przy dodatkowym założeniu twierdzenie Abela. Zostało udowodnione przez słowackiego matematyka Alfreda Taubera. 

\section*{Sformułowanie}

Niech \(f(z)=\sum _{n=0}^{\infty }a_{n}z^{n}\) będzie szeregiem potęgowym o promieniu zbieżności 1. Jeśli na zbiega do zera przy n an dążącym do nieskończoności oraz dla pewnego ciągu $(z_{n})_{n\in \mathbb {N} }\in \mathbb {C} ^{\mathbb {N} }$ o wyrazach spełniających dla pewnego K warunek: ${\frac {|1-z_{n}|}{1-|z_{n}|}}<K$ zachodzi f(zn) → s, to szereg jest zbieżny i \(\sum _{n=0}^{\infty }a_{n}=s\).

\section*{Dowód}
Oznaczmy przez N(z) liczbę całkowitą taką, że:

\begin{displaymath}
	N(z)\leq {\frac {1}{1-|z|}}<N(z)+1
\end{displaymath}

Przy zn dążących do N(zn) → nieskończoność. Zatem ponieważ

\begin{equation*}
	\left|\sum _{n=N(z_{k})}^{N(z_{l})}a_{n}\right|\leq \left|\sum _{n=0}^{N(z_{l})}a_{n}-f(z_{l})\right|+|f(z_{l})-f(z_{k})|+\left|f(z_{k})-\sum _{n=0}^{N(z_{k})}a_{n}\right|
\end{equation*}

jeśli f(zn) spełnia warunek Cauchy'ego, to by wykazać że suma an też do spełnia wystarczy udowodnić, że \(f(z)-\sum _{n=0}^{N(z)}a_{n}\) dąży do zera.

Z założenia dla dostatecznie dużych N dla wszystkich n > N zachodzi |n an| jest mniejsze od epsilon, więc wówczas: 

\[\left|\sum _{n=N(z)+1}^{\infty }a_{n}z^{n}\right|=\left|\sum _{n=N(z)+1}^{\infty }na_{n}{\frac {z^{n}}{n}}\right|<{\frac {\epsilon }{N(z)+1}}\sum _{n=N(z)+1}^{\infty }|z|^{n}<\epsilon {\frac {\frac {1}{1-|z|}}{N(z)+1}}<\epsilon\]

Ponieważ

\begin{equation}
	|1-z^{n}|=\left|(1-z)\left(\sum _{k=0}^{n}z^{k}\right)\right|<n|1-z|
\end{equation}

zaś z twierdzenia o zbieżności średnich wynika, że 

\begin{displaymath}
	\left|f(z_{n})-\sum _{k=0}^{N(z_{n})}a_{k}\right|\leq \left|\sum _{k=N(z_{n})+1}^{\infty }a_{k}z^{k}\right|+\left|-\sum _{k=0}^{N(z_{n})}a_{k}(1-z^{k})\right|<\epsilon +\epsilon
\end{displaymath}

Uwaga: z twierdzenia Abela wynika, że zawsze można wziąć ciąg należący do odcinka (0 1), bo jeśli szereg jest zbieżny, to zbieżność f(zn) zachodzi dla każdego ciągu spełniającego warunek.



\end{document}