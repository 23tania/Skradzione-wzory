\documentclass{article}
\usepackage[utf8]{inputenc}
\usepackage[MeX,plmath]{polski} 
\usepackage{amsmath}
\usepackage{amsfonts}

\begin{document}
\title{Szereg Fouriera - lematy}
\maketitle

\section*{Szereg Fouriera - przypomnienie}
Trygonometrycznym szeregiem Fouriera funkcji f nazywamy szereg funkcyjny następującej postaci: 
\begin{gather*}
S(x)={\frac {a_{0}}{2}}+\sum _{n=1}^{\infty }\left(a_{n}\cos \left({\frac {2n\pi }{T}}x\right)+b_{n}\sin \left({\frac {2n\pi }{T}}x\right)\right)
\end{gather*}

\section*{Lemat I}
\begin{itemize}
\item n jest liczbą całkowitą
\[ \int \limits _{-T}^{T}\cos n\omega xdx=\left\{{\begin{aligned}&0&&{\text{gdy}}~~n\neq 0\\&2T&&{\text{gdy}}~~n=0\end{aligned}}\right. \]
\[ \int \limits _{-T}^{T}\sin n\omega xdx=0 \]

\item m, n są liczbami naturalnymi
\[ \int \limits _{-T}^{T}\cos n\omega x\cos m\omega xdx=\left\{{\begin{aligned}&0&&{\text{gdy}}~~n\neq m\\&T&&{\text{gdy}}~~n=m\end{aligned}}\right. \]
\[ \int \limits _{-T}^{T}\sin n\omega x\sin m\omega xdx=\left\{{\begin{aligned}&0&&{\text{gdy}}~~n\neq m\\&T&&{\text{gdy}}~~n=m\end{aligned}}\right. \]
\[ \int \limits _{-T}^{T}\sin n\omega x\cos m\omega xdx=0 \]
\end{itemize}

\section*{Lemat II}
$$ {\frac {1}{2}}+\sum _{n=1}^{N}\cos n\alpha ={\frac {\sin \left(N+{\frac {1}{2}}\right)\alpha }{2\sin {\frac {1}{2}}\alpha }} $$

\subsubsection*{Dowód}
\begin{displaymath}
{\begin{aligned}&\;{\frac {1}{2}}+\sum _{n=1}^{N}\cos n\alpha \\=&\ \Re \left(-{\frac {1}{2}}+\sum _{n=0}^{N}e^{in\alpha }\right)\\=&\ \Re \left(-{\frac {1}{2}}+{\frac {1-e^{i(N+1)\alpha }}{1-e^{i\alpha }}}\right)\\=&-{\frac {1}{2}}+{\frac {1}{|1-e^{i\alpha }|^{2}}}\Re ((1-e^{i(N+1)\alpha })(1-e^{-i\alpha }))\\=&-{\frac {1}{2}}+{\frac {1}{(1-\cos \alpha )^{2}+\sin ^{2}\alpha }}\Re (1-e^{-i\alpha }-e^{i(N+1)\alpha }+e^{iN\alpha })\end{aligned}}
\end{displaymath}
więc mamy (biorąc cześć rzeczywistą i stosując podstawowe wzory trygonometryczne): 
\begin{displaymath}
{\begin{aligned}&{\frac {1}{2}}+\sum _{n=1}^{N}\cos n\alpha \\=&-{\frac {1}{2}}+{\frac {1-\cos \alpha -\cos(N+1)\alpha +\cos N\alpha }{2(1-\cos \alpha )}}\\=&\;{\frac {\cos N\alpha -\cos(N+1)\alpha }{2(1-\cos \alpha )}}\\=&\;{\frac {2\sin \left(N+{\frac {1}{2}}\right)\alpha \sin {\frac {1}{2}}\alpha }{4\sin ^{2}{\frac {1}{2}}\alpha }}\\=&\;{\frac {\sin \left(N+{\frac {1}{2}}\right)\alpha }{2\sin {\frac {1}{2}}\alpha }}\end{aligned}}
\end{displaymath}
q. e. d.

\section*{Lemat III}
Jeżeli f jest funkcją ciągłą w przedziale [a, b] z wyjątkiem co najwyżej skończonej ilości punktów i bezwzględnie całkowalną w tym przedziale to 
\begin{equation*}
\lim _{n\to \infty }\int \limits _{a}^{b}f(x)\sin nx\,{\text{d}}x=0
\end{equation*}


\end{document}