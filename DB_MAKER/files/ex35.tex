\documentclass{article}
\usepackage[utf8]{inputenc}
\usepackage[MeX,plmath]{polski} 
\usepackage{amsmath}
\usepackage{amsfonts}

\begin{document}
\title{Twierdzenie Cauchy’ego (rachunek różniczkowy)}
\maketitle

\section*{Wstęp}
Twierdzenie Cauchy’ego – twierdzenie teorii grup, mówi ono, że jeśli G jest grupą skończoną i p jest liczbą pierwszą, będącą dzielnikiem rzędu grupy G (liczby elementów grupy G), to w G istnieje element rzędu p. Oznacza to, że istnieje \(x\in G\) taki, że dla najmniejszego niezerowego zachodzi $x^p = e$, gdzie e jest elementem neutralnym.

Powyższe twierdzenie związane jest z twierdzeniem Lagrange’a, które mówi, że rząd dowolnej skończonej podgrupy grupy G dzieli rząd grupy G. Z twierdzenia Cauchy’ego wynika, że dla dowolnej liczby pierwszej p będącej dzielnikiem rzędu G, istnieje podgrupa grupy której rzędem jest p i jest to grupa cykliczna.

Twierdzenie Cauchy’ego jest uogólnione przez pierwsze twierdzenie Sylowa, które zakłada, że jeśli p jest liczbą pierwszą, a p do n jest dzielnikiem rzędu grupy G, to G ma podgrupę rzędu p do n.


\section*{Twierdzenie}

Jeżeli dane funkcje f i g są: \\
- ciągłe w przedziale domkniętym [a,b], \\
- różniczkowalne w przedziale (a, b),

to istnieje punkt c należący do przedziału (a,b) taki, że: 

\begin{equation}
	g'(c)\cdot \left[f(b)-f(a)\right]=f'(c)\cdot \left[g(b)-g(a)\right]
\end{equation}



\section*{Dowód}

Nasza grupa cykliczna działa na zbiór 

\begin{equation*}
	X=\{(x_{1},\dots ,x_{p})\in G^{p}:x_{1}x_{2}\ldots x_{p}=e\}
\end{equation*} 

skończonych ciągów z G o długości których iloczyn daje element neutralny. Zbiór tych p elementów jest jednoznacznie określony przez wszystkie jego składniki z wyjątkiem ostatniego, ostatni element musi być odwrotnością iloczynu poprzednich elementów.

Zdefiniujmy h: [a,b] → R

\begin{equation*}
	h(x)=(f(b)-f(a))g(x)-(g(b)-g(a))f(x)
\end{equation*}

Zauważmy, że h jest różniczkowalna na (a,b) oraz h(a)=h(b), więc na mocy twierdzenia Rolle’a istnieje c należące do przedziału (a,b) takie, że h'(c)=0. Ponadto 

\begin{equation}
	0=h'(c)=(f(b)-f(a))g'(c)-(g(b)-g(a))f'(c)
\end{equation}

co kończy dowód.

\section*{Wniosek}
Jeżeli funkcje f i g są:

- ciągłe w przedziale domkniętym [a,b], różniczkowalne w przedziale (a,b) oraz dodatkowo $g'(x)\not =0$ dla \(x\in (a,b)\),

to istnieje taki punkt $c\in (a,b)$, że: 

\begin{displaymath}
	{\frac {f(b)-f(a)}{g(b)-g(a)}}={\frac {f'(c)}{g'(c)}} 
\end{displaymath}

\section*{Zastosowania}
Natychmiastową konsekwencją twierdzenia Cauchy’ego jest charakteryzacja p-grup skończonych, gdzie p jest liczbą pierwszą. W szczególności, skończona grupa G jest p-grupą (czyli wszystkie jej elementy mają rząd p do k dla dowolnej liczby naturalnej k) wtedy i tylko wtedy, gdy G ma rząd p do n dla dowolnej liczby naturalnej n. Możemy skorzystać z przypadku abelowego twierdzenia Cauchy’ego w dowodzie indukcyjnym pierwszego twierdzenia Sylowa podobnie jak w pierwszym dowodzie powyżej, chociaż istnieją dowody, w których unikamy sprawdzania osobno specjalnego przypadku.




\end{document}