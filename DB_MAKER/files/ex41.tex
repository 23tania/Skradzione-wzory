\documentclass{article}
\usepackage[utf8]{inputenc}
\usepackage[MeX,plmath]{polski} 
\usepackage{amsmath}
\usepackage{amsfonts}

\begin{document}
\title{Twierdzenie Phragména-Lindelöfa}
\maketitle

\section*{Twierdzenie}
Twierdzenie Phragména-Lindelöfa – niech dana będzie funkcja ciągła f(z) o argumentach zespolonych oraz ograniczona dla argumentów zawartych w przedziale \(\alpha \leq a\leq \beta\) i holomorficzna wewnątrz tegoż przedziału. Jeśli dla a = alfa i a = beta istnieje takie M, że zachodzi

\[\forall _{z\in <\alpha ,\beta >}\colon |f(z)|\leq M\]

Jeżeli ponadto $\exists _{q\in <\alpha ,\beta >}\colon |f(q)|\ =\ M$ to f jest funkcją stałą.

\section*{Dowód}
Załóżmy, że $f(a\ +\ b\cdot i)\to 0$ jednostajnie dla b dążącego do +- nieskończoności dla z należącego do alfa,beta. Niech

\begin{displaymath}
	q\ =\ a_{q}\ +\ b_{q}\cdot i\in <\alpha ,\beta >
\end{displaymath}

Wtedy

\begin{equation*}
	\exists _{m>|b_{q}|}\forall _{|b|\geq m}\exists _{a\in <\alpha ,\beta >}\colon |f(a\ +\ b\cdot i)|\leq M
\end{equation*}

Niech P będzie wnętrzem prostokąta wyznaczonego przez zbiór:

\[Pr\ =\ \{a\ +\ b\cdot i\ :\ \alpha \leq a\leq \beta \ \land \ -m\leq b\leq m\}\]

Jeżeli funkcja f jest stała, to twierdzenie jest w oczywisty sposób prawdziwe. W przeciwnym przypadku f nie jest stała w alfa, beta, wtedy nie może być stała w P, i na podstawie zasady maksimum f nie osiąga kresu górnego w P. Ponieważ |f(z)| jest ciągła w Pr, to |f(z)| osiąga swój kres górny w Pr. Punkt w którym osiąga ona swój kres górny, nie może należeć do P, a ponieważ |f(z)| jest mniejsze lub równe od M na bokach prostokąta Pr, więc |f(z)| jest mniejsze od M w P, w szczególności |f(q)| jest mniejsze od M.

Niech:

\begin{equation}
	f_{n}(z)\ =\ f(z)\cdot \exp ^{z^{\frac {2}{n}}}\ =\ f(z)\cdot \exp ^{\frac {a^{2}+b^{2}}{n}}\cdot \exp ^{\frac {2\cdot i\cdot a\cdot b}{n}}
\end{equation}

dla n = 1,2,... \\

będzie funkcją. Jest ona ciągła oraz ograniczona i holomorficzna w alfa, beta. Dodatkowo dla x = alfa i x = beta zachodzi: 

\begin{displaymath}
	|f_{n}(z)|\ =\ |f(z)|\cdot \exp ^{\frac {a^{2}+b^{2}}{n}}\leq |f(z)|\cdot \exp ^{\frac {\delta ^{2}}{n}}
\end{displaymath}

dla \(\delta \ =\ \max(|\alpha |,|\beta |)\)


\end{document}