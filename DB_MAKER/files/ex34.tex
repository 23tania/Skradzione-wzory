\documentclass{article}
\usepackage[utf8]{inputenc}
\usepackage[MeX,plmath]{polski} 
\usepackage{amsmath}
\usepackage{amsfonts}

\begin{document}
\title{Twierdzenie Darboux}
\maketitle

\section*{Wstęp}
Twierdzenie Darboux – twierdzenie analizy rzeczywistej noszące nazwisko Jeana Darboux, które zapewnia o tym, że każda rzeczywista funkcja ciągła ma własność Darboux; w szczególności: każda funkcja ciągła określona na przedziale rzeczywistym przyjmuje wszystkie wartości pośrednie między obrazami krańców przedziału. Stąd pochodzi inna nazwa twierdzenia, mianowicie twierdzenie o przyjmowaniu wartości pośrednich lub krócej twierdzenie o wartości pośredniej; z twierdzeniem wiążą się również nazwiska Bernarda Bolzana i Augustina Louisa Cauchy’ego (nazwy twierdzenie Bolzana-Cauchy’ego lub twierdzenie Cauchy’ego nie zdobyły popularności w polskiej literaturze matematycznej).

\section*{Twierdzenie}

Niech f:[a,b] → R będzie funkcją ciągłą. Jeżeli \(f(a)\cdot f(b)<0\) (tzn. wartości funkcji f na końcach przedziałów mają różne znaki), to istnieje taki punkt c w przedziale (a, b), dla którego

\begin{equation}
	f(c)=0
\end{equation}

Ogólniej: każda funkcja ciągła f : [ a , b ] → R ma własność Darboux, tzn. jeśli d spełnia jedną z nierówności \(f(a)<d<f(b)\) lub f(a) większe od d większe od f(b), to istnieje taki punkt c w przedziale [a,b], dla którego

\begin{displaymath}
	f(c)=d
\end{displaymath}

Oba sformułowania są równoważne: funkcje f w obu z nich różnią się jedynie o stałą d.


\section*{Dowody}

\subsection*{Analityczny z definicji Cauchy'ego ciągłości}

Niech f:[a,b] → R. Bez straty ogólności można założyć, że d jest liczbą z przedziału otwartego (f(a),f(b)). 

Niech

\begin{equation*}
	A=\{x\in [a,b]\colon f(x)\leq d\}=f^{-1}[(-\infty ,d]]
\end{equation*}

\begin{equation}
	A^{c}=\{x\in [a,b]\colon f(x)>d\}=f^{-1}[(d,+\infty )]
\end{equation}

Wówczas zbiory A i A do c są niepuste. Zbiór A posiada ograniczenie górne, którym jest b więc istnieje na mocy aksjomatu ciągłości s = sup A. Dla danych T należy do R , p0 należy do T oraz r większe od 0 oznaczmy 

\[B_{T}(p_{0};r)=\{p\in T:|p-p_{0}|<r\}\]

Wykażemy, że f(s)=d. Istotnie, wobec ciągłości funkcji, właściwości supremum oraz rozłączności zbiorów A i A do c spełnione są następujące ciągi implikacji:

\begin{displaymath}
	f(s)<d\Rightarrow (s\neq b)\,\land \,\exists {\delta >0}\;(B_{[a,b]}(s;\delta )\subset f^{-1}[B_{\mathbb {R} }(f(s);d-f(s))]\subset A)\,\land \, 
\end{displaymath}
\[(s+\delta <b)\Rightarrow ((s,s+\delta )\subset [a,b])\,\land \,((s,s+\delta )\cap A\neq \emptyset )\Rightarrow \sup A\neq s\]

\begin{equation*}
	f(s)>d\Rightarrow \exists {\delta >0}\;B_{[a,b]}(s;\delta )\subset f^{-1}[B_{\mathbb {R} }(f(s);f(s)-d)]\subset A^{c}\Rightarrow A\cap (s-\delta ,s]=\emptyset \Rightarrow \sup A\neq s
\end{equation*}

Zatem poprzez sprzeczność dowodzi się, że nie jest możliwym aby f(s) było różne od d.

\subsection*{Topologiczny}
Niech f:[a,b] → R będzie funkcją oraz niech d będzie liczbą z przedziału otwartego (f(a),f(b)). Przypuśćmy, że d nie jest wartością funkcji f. Wówczas przeciwobraz przestrzeni topologicznej R \ {d} powinien być równy dziedzinie (którą tutaj jest przedział [a,b], jednak wobec ciągłości funkcji będzie on sumą dwóch niepustych, rozłącznych, otwartych przeciwobrazów, a zatem przestrzenią niespójną, co wyklucza się z faktem spójności drogowej dziedziny. Wobec czego poprzez sprzeczność dowodzi się, że d nie może nie być wartością funkcji. 

Przedstawione rozumowanie wiąże się z twierdzeniem o szerszym zakresie mówiącym, że ciągły obraz zbioru spójnego jest zbiorem spójnym.

\end{document}