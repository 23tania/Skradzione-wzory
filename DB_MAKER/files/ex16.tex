\documentclass{article}
\usepackage[utf8]{inputenc}
\usepackage[MeX,plmath]{polski} 
\usepackage{amsmath}
\usepackage{amsfonts}

\begin{document}
\title{Metody Newtona-Cotesa}
\maketitle

\section*{Wprowadzenie}
Metody Newtona-Cotesa – zbiór metod numerycznych całkowania, zwanego również kwadraturą. Nazwa pochodzi od Isaaca Newtona i Rogera Cotesa.

Przyjmujemy, że wartości funkcji f(x) są znane w równo oddalonych punktach (węzłach) \( x_{i} dla i = 0 , . . . , n \). Dla węzłów nierówno oddalonych od siebie maja zastosowanie inne wzory np. kwadratura gaussowska.

Jeżeli $ a=x_{0}<x_{1}<x_{2},\dots <x_{n-1}<x_{n}=b $ są równoodległymi węzłami interpolacji funkcji f(x), to całkę: 
$$ \int \limits _{a}^{b}f(x)dx $$
można aproksymować całką: 
\begin{displaymath}
\int \limits _{a}^{b}L_{n}(x)dx
\end{displaymath} 
gdzie Ln(x)dx jest wielomianem interpolacyjnym Lagrange’a stopnia co najwyżej n, przybliżającym funkcję f(x) w węzłach interpolacji, tj.:
\begin{math} L_{n}(x_{0})=y(x_{0}),L_{n}(x_{1})=y(x_{1}),\dots ,L_{n}(x_{n})=y(x_{n}) \end{math}\\

Niech $ h_{n}={\frac {b-a}{n}} $ oznacza długość kroku dzielącą dwa węzły interpolacji. 

Można zapisać:
\begin{equation*}
\lambda _{i}(x)=\lambda _{i}(a+th)=\prod _{j=0\land j\neq i}^{n}{\frac {t-j}{i-j}}=g(t)
\end{equation*}

Wtedy:
\begin{gather*}
\int \limits _{a}^{b}L_{n}(x)dx=\int \limits _{a}^{b}\sum _{i=0}^{n}f(x_{i})\cdot \lambda _{i}(x)dx=\sum _{i=0}^{n}f(x_{i})\cdot \int \limits _{a}^{b}\lambda _{i}(x)dx
\end{gather*}

Zmieniając zmienną, oraz granice całkowania otrzymuje się: 
\[ \int \limits _{a}^{b}L_{i}(x)dx=h\cdot \int \limits _{0}^{n}g(t)dt \]

Ostatecznie, wzór Newtona-Cotesa dla n+1 równo odległych węzłów przyjmuje postać: 
\begin{multline*}
\int \limits _{a}^{b}f(x)dx=\int \limits _{a}^{b}L_{n}(x)dx=\sum _{i=0}^{n}f(x_{i})\cdot \int \limits _{a}^{b}\lambda _{i}(x=a+t\cdot h)dx=\\\sum _{i=0}^{n}f(x_{i})\cdot h\cdot \int \limits _{0}^{n}\prod _{j=0\land j\neq i}^{n}{\frac {t-j}{i-j}}dt
\end{multline*}
\end{document}