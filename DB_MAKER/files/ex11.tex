\documentclass{article}
\usepackage[utf8]{inputenc}

\begin{document}
\title{Wszystkie rozwiązania: wzory Cardana}
\maketitle

\section{Wszystkie rozwiązania: wzory Cardana}
Poniżej będzie przedstawiona metoda, pozwalająca otrzymać wszystkie pierwiastki równania (2), jeśli jeden został już znaleziony według powyższej metody. Niech ε 0 , ε 1 , ε 2  będą pierwiastkami 3. stopnia z jedynki, tzn. 
$\varepsilon_0=1,$ $\varepsilon_1=\frac{-1 + i\sqrt 3}{2},$ $\varepsilon_2= \frac{-1 - i\sqrt 3}{2}.$

Tak jak wcześniej, niech z 0 będzie pierwiastkiem równania (6): 
$$z_0=\frac{-q+\sqrt{q^2+4p^3/27}}{2}.$$
Ustalmy liczby v 0 , u ∗ takie, że 
$(v_0)^3=z_0$ oraz $(u_*)^3=-q-z_0=\frac{-q-\sqrt{q^2+4p^3/27}}{2}$
Zauważmy, że 
$$(v_0)^3\cdot (u_*)^3=\frac{-q+\sqrt{q^2+4p^3/27}}{2}\cdot \frac{-q-\sqrt{q^2+4p^3/27}}{2}=\frac{q^2-q^2-4p^3/27}{4}=\frac{-p^3}{27}.$$
Zatem dla pewnego m ∈ { 0 , 1 , 2 }mamy że:
$$v_0\cdot u_*=-\varepsilon_m p/3.$$
\end{document}