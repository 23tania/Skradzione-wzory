\documentclass{article}
\usepackage[utf8]{inputenc}
\usepackage[MeX,plmath]{polski} 
\usepackage{amsmath}
\usepackage{amsfonts}

\begin{document}
\title{Całki eliptyczne}
\maketitle

\section*{Wprowadzenie}
Całki eliptyczne – to ważna klasa całek postaci
\begin{equation}
\int R(x,{\sqrt {W(x)}})dx
\end{equation}
gdzie R(x,y) jest funkcją wymierną zmiennych x i y, a W(x) jest wielomianem o współczynnikach rzeczywistych stopnia 3 lub 4.

\section*{Nazwa całek eliptycznych}
Z całkami eliptycznymi po raz pierwszy zetknięto się podczas obliczania obwodu elipsy, stąd też wzięły swoją nazwę. Nazwa ich nie jest jednak jednoznaczna, ponieważ w ścisłym znaczeniu dotyczy tylko tych całek postaci (1), które nie dają się wyrazić za pomocą funkcji elementarnych. Te z nich, które sprowadzają się do postaci skończonej, nazywa się całkami pseudoeliptycznymi. 

\section*{Rodzaje całek eliptycznych}
Choć całki postaci (1) nie wyrażają się zwykle przez funkcje elementarne, to każdą z nich można za pomocą podstawień doprowadzić do jednej z następujących trzech całek 
\begin{itemize}
\item $ \int {\frac {dt}{\sqrt {(1-t^{2})(1-k^{2}t^{2})}}}\ \ (0<k<1) $
\item $ \int {\frac {(1-k^{2}t^{2})dt}{\sqrt {(1-t^{2})(1-k^{2}t^{2})}}}\ \ (0<k<1) $
\item $ \int {\frac {dt}{(1+ht^{2}){\sqrt {(1-t^{2})(1-k^{2}t^{2})}}}}\ \ (0<k<1) $
\end{itemize}

Całek tych, jak pokazał Liouville, nie da już wyrazić się za pomocą funkcji elementarnych. 

Legendre zastosował podstawienie t = sin(phi) dzięki czemu całki te uprościły swoją postać do całek, które nazywamy odpowiednio całką eliptyczną pierwszego, drugiego i trzeciego rodzaju w postaci Legendre’a, tj. 
\begin{itemize}
\item \( \int {\frac {d\phi }{\sqrt {(1-k^{2}\sin ^{2}\phi )}}}\ \ (0<k<1) \) - całka eliptyczna 1. rodzaju
\item \( \int {\sqrt {(1-k^{2}\sin ^{2}\phi )}}\;d\phi \ \ (0<k<1) \) - całka eliptyczna 2. rodzaju
\item \( \int {\frac {d\phi }{(1+h\sin ^{2}\phi ){\sqrt {(1-k^{2}\sin ^{2}\phi )}}}}\ \ (0<k<1) \) - całka eliptyczna 3. rodzaju
\end{itemize}

Szczególnie ważne i często używane są pierwsze dwie z nich. 

\section*{Obliczanie obwodu elipsy}
Dokładną wartość obwodu elipsy wyznacza całka eliptyczna zupełna drugiego rodzaju, wzorem
$$ l=4a\,{\text{E}}{\big (}e{\big )} $$
gdzie $ e={\sqrt {a^{2}-b^{2}}}/a={\sqrt {1-(b/a)^{2}}} $ - mimośród elipsy.

\section*{Funkcje odwrotne do całek eliptycznych}
Funkcjami odwrotnymi do całek eliptycznych są funkcje eliptyczne. Na przykład funkcja eliptyczna Weierstrassa zmiennej zespolonej z jest funkcją odwrotną do funkcji wyrażonej przez całkę 
\[ z(w)=\int \limits _{w}^{\infty }{\frac {dt}{\sqrt {4t^{3}-g_{2}t-g_{3}}}} \]

Funkcjami odwrotnymi do całek eliptycznych są funkcje amplitudy. 

\end{document}