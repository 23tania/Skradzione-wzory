\documentclass{article}
\usepackage[utf8]{inputenc}
\usepackage[MeX,plmath]{polski} 
\usepackage{amsmath}
\usepackage{amsfonts}

\begin{document}
\title{Równanie różniczkowe Poissona}
\maketitle

\section*{Wprowadzenie}
Równanie różniczkowe Poissona – niejednorodne równanie różniczkowe cząstkowe liniowe drugiego rzędu typu eliptycznego.

Równanie to zapisać można w postaci: 
\[ \nabla^{2}u=f \]
lub inaczej
$$ \triangle u = f $$

Funkcję f zmiennych przestrzennych traktuje się jako znaną.

Równanie można również zapisać explicite dla przestrzeni o zadanym wymiarze.

Dla przestrzeni trójwymiarowej przyjmuje ono postać równania różniczkowego cząstkowego: 
\begin{displaymath}
{\frac {\partial ^{2}}{\partial x^{2}}}u(x,y,z)+{\frac {\partial ^{2}}{\partial y^{2}}}u(x,y,z)+{\frac {\partial ^{2}}{\partial z^{2}}}u(x,y,z)=f(x,y,z)
\end{displaymath}
a dla dwuwymiarowej:
\begin{equation*}
{\frac {\partial ^{2}}{\partial x^{2}}}u(x,y)+{\frac {\partial ^{2}}{\partial y^{2}}}u(x,y)=f(x,y)
\end{equation*}

W przypadku jednowymiarowym równanie Poissona redukuje się do równania różniczkowego zwyczajnego: 
$$ u''(x)=f(x) $$

W przypadku jednorodnym, tj. jeśli $ f\equiv 0 $, to mamy do czynienia z przypadkiem szczególnym znanym pod nazwą równania różniczkowego Laplace’a. 

Równanie Poissona opisuje wiele procesów zachodzących w przyrodzie, np. rozkład pola prędkości cieczy wypływającej ze źródła, potencjał pola grawitacyjnego w obecności źródeł, potencjał pola elektrostatycznego w obecności ładunków, temperaturę wewnątrz ciała przy stałym dopływie ciepła.

Nazwa równania pochodzi od nazwiska Simeona Denisa Poissona, który sformułował je na początku XIX wieku i przeprowadził analizę jego rozwiązań. 

\section*{Rozwiązania i funkcje Greena}
Równanie różniczkowe Poissona z dołączonymi do niego warunkami brzegowymi tworzy eliptyczne zagadnienie brzegowe. Zagadnienie to posiada rozwiązania regularne, o ile warunki brzegowe mają postać ciągłą. 
\begin{gather*}
u(x)=\int _{\partial U}g(y){\frac {\partial G}{\partial n}}(x,y)dS(x,y)+\int _{U}f(y)G(x,y)dy
\end{gather*}
gdzie G jest funkcją Greena obszaru (o ile dla danego obszaru taka funkcja istnieje). 

Funkcją Greena półprzestrzeni \begin{math} \mathbb {R} _{+}^{n}=\{x=(x_{1},\dots ,x_{n}):x_{n}>0\} \end{math} jest 
\[ G(x,y)=\Gamma (y-x)-\Gamma (y-{\bar {x}}) \]

Funkcją Greena (hiper)kuli jest
$$ G(x,y)=\Gamma (y-x)-\Gamma (|x|(y-{\tilde {x}})) $$

\end{document}