\documentclass{article}
\usepackage[utf8]{inputenc}
\usepackage[MeX,plmath]{polski} 
\usepackage{amsmath}
\usepackage{amsfonts}

\begin{document}
\title{Równanie falowe}
\maketitle

\section*{Wprowadzenie}
Równanie falowe – matematyczne równanie różniczkowe cząstkowe drugiego rzędu opisujące ruch falowy. 

Ogólną postacią równania falowego jest: 
\begin{displaymath}
{\begin{cases}{\frac {\partial {}^{2}}{\partial t^{2}}}u-c^{2}\cdot \triangle _{x}u=0,&u:\mathbb {R} ^{n}\times \mathbb {R} _{+}\to {}\mathbb {R} ,x\in \mathbb {R} ^{n},t\in \mathbb {R} _{+}\\{[2pt]}u(x,0)=f(x),&f:\mathbb {R} ^{n}\to \mathbb {R} \\{[2pt]}{\frac {\partial {}}{\partial t}}u(x,0)=g(x),&g:\mathbb {R} ^{n}\to \mathbb {R} \end{cases}}
\end{displaymath}

Skrótowo można wyrazić równanie falowe używając operatora d’Alemberta: 
$$ \square u(x,t)=0 $$

Rozwiązania równania falowego mają różne postaci i własności w zależności od parzystości wymiaru przestrzeni. Najważniejsze równania falowe to przypadki n = 1, 2, 3.

Równanie falowe jest ważne w mechanice kwantowej, gdyż opisuje falę de Broglie’a: 
\[ e^{i(Et-{\vec {r}}\circ {\vec {p}})/\hbar } \]

Równanie falowe można wyprowadzić z równań Maxwella. 

\section*{Rozwiązania równania falowego}
\subsection*{Równanie struny i wzór d'Alemberta}
Jednowymiarowe (n = 1) równanie falowe nazywa się równaniem struny lub równaniem fali płaskiej. Ma ono postać: 
\begin{gather*}
{\begin{cases}{\frac {\partial {}^{2}}{\partial t^{2}}}u-c^{2}\cdot \triangle _{x}u=0,&u:\mathbb {R} \times \mathbb {R} _{+}\to {}\mathbb {R} \\u(x,0)=f(x),&f:\mathbb {R} \to \mathbb {R} \\{[2pt]}{\frac {\partial {}}{\partial t}}u(x,0)=g(x),&g:\mathbb {R} \to \mathbb {R} \end{cases}}
\end{gather*}

Bez uwzględnienia warunków brzegowych rozwiązaniem jest: 
$$ u(x,t)=\alpha (x-ct)+\beta (x+ct) $$

Przy założeniu regularności oraz uwzględnieniu warunku brzegowego rozwiązaniem jest: 
\[ u(x,t)={\frac {f(x+ct)+f(x-ct)}{2}}+{\frac {1}{2c}}\int \limits _{x-ct}^{x+ct}{g(z)dz} \]

Jest to wzór d’Alemberta. Równanie struny jest wówczas poprawnie postawione. 

\subsection*{Równanie struny półnieskończonej}
Struna półnieskończona to jednowymiarowa struna przymocowana na stałe z jednego końca. Matematycznie odpowiada dodaniu dodatkowego warunku brzegowego: 
$$ u(0,t)=0 {\text{ dla  dowolnego }} t\in \mathbb {R} $$

Rozwiązaniem zagadnienia struny półnieskończonej jest: 
\begin{equation*}
{\begin{cases}u(x,t)={\frac {f(x+ct)+f(x-ct)}{2}}+{\frac {1}{2c}}\int \limits _{x-ct}^{x+ct}{g(z)dz},&x\geq {}ct\\{[2pt]}u(x,t)={\frac {f(x+ct)-f(ct-x)}{2}}+{\frac {1}{2c}}\int \limits _{ct-x}^{ct+x}{g(z)dz},&x<ct\end{cases}}
\end{equation*}

\subsection*{Równanie falowe w wymiarze 3 i wzór Kirchhoffa}
Równanie falowe dla n = 3 ma postać
\[ {\begin{cases}{\frac {\partial {}^{2}}{\partial t^{2}}}u-c^{2}\cdot \triangle _{x}u=0,&u:\mathbb {R} ^{3}\times \mathbb {R} _{+}\to {}\mathbb {R} \\u(x,0)=f(x),&f:\mathbb {R} ^{3}\to \mathbb {R} \\{[2pt]}{\frac {\partial {}}{\partial t}}u(x,0)=g(x),&g:\mathbb {R} ^{3}\to \mathbb {R} \end{cases}} \]

Rozwiązanie równania można wyprowadzić za pomocą średnich sferycznych. 
$$ 4\pi {}c^{2}{}\cdot {}u(x,t)={\frac {\partial }{\partial t}}\left({\frac {1}{t}}\int \limits _{S^{2}(x,ct)}{f(z)d\sigma (z)}\right)+{\frac {1}{t}}\int \limits _{S^{2}(x,ct)}{g(z)d\sigma (z)} $$

Jest to wzór Kirchhoffa. 

\end{document}