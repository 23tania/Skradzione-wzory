\documentclass{article}
\usepackage[utf8]{inputenc}

\begin{document}
\title{Metoda wariacyjna}
\maketitle

\section{Metoda wariacyjna}
W porównaniu z rachunkiem zaburzeń, metoda wariacyjna ma pewną przewagę – może ona być użyta praktycznie do dowolnego układu, nie trzeba na nią nakładać żadnych dodatkowych ograniczeń. Równanie Schrödingera przedstawia się następująco: 
$${ {\hat {H}}\psi _{n}(r)=E_{n}\psi _{n}(r).}$$

Nie można go rozwiązać ściśle, jednak można znaleźć jego przybliżone funkcje i wartości własne. W stanie podstawowym energię można oznaczyć jako E 0 , czyli:
$${ E_{0}<E_{n}~(n>0).}$$

Można teraz założyć, że istnieje pewna funkcja φ  w tej samej przestrzeni co ψ n  i za jej pomocą można zdefiniować parametr ϵ :
$${ \epsilon ={\frac {\int \varphi ^{*}{\hat {H}}\varphi d\tau }{\int \varphi ^{*}\varphi d\tau }}.}$$

Ponieważ funkcje ψ n tworzą układ zupełny funkcji ortonormalnych, funkcję φ  można przedstawić w postaci szeregu:

$${ \varphi =\sum \limits _{n}c_{n}^{*}\psi _{n}.}$$

Jeżeli funkcja φ jest także znormalizowana, to powyższe równania można przedstawić w postaci: 
$${ \sum \limits _{n}c_{n}^{*}c_{n}=1,}$$

a zatem parametr ϵ będzie miał postać: 
$${ \epsilon =\sum \limits _{n}c_{n}^{*}c_{n}E_{n}.}$$

Jeśli od obu stron równania odjąć wartość E otrzyma się: 
$${ \epsilon -E_{0}=\sum \limits _{n}c_{n}^{*}c_{n}(E_{n}-E_{0}).}$$

Wobec zawsze dodatniej prawej strony równania (iloczyn c n ∗ oraz różnica energii są zawsze dodatnie), lewa strona równania także jest dodatnia. Skoro:
$${ \epsilon -E_{0}\geqslant 0,}$$
to: 
$${ \epsilon \geqslant E_{0}.}$$
Wynik ten w połączeniu ze wzorem jest podstawą metody wariacyjnej. 
$${ \varphi =\varphi (a_{1},a_{2},\dots ,a_{k};r_{1},r_{2},\dots ,r_{n}).}$$
\end{document}