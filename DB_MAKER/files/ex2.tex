\documentclass{article}
\usepackage[utf8]{inputenc}
\usepackage[MeX,plmath]{polski} 

\begin{document}
\section{Introduction}
Lorem ipsum dolor sit amet tempor magna. Sed nonummy sagittis. Lorem ipsum primis in sem. Quisque ut nulla non dolor. Vestibulum convallis ligula nunc, placerat ut, eleifend ac, mollis orci. Integer eu euismod non, tempor varius, leo. Donec vitae velit wisi, posuere sed, suscipit enim. Phasellus pulvinar mollis, orci luctus at, metus. Nullam vulputate vehicula. Vivamus imperdiet in, ante. Quisque arcu massa, dictum eget, neque. Nam convallis diam ut tortor lacus vehicula sit amet, ante. Quisque iaculis, diam eu lacus. Aliquam tellus hendrerit et, neque. Donec non nunc. Vivamus orci imperdiet ut, tempus enim. Phasellus blandit, quam. Mauris imperdiet orci.\\
\begin{math}
    \int {f_N(x) \stackrel{=} 1}
\end{math}
Litwo! Ojczyzno moja! Ty jesteś jak zdrowie. Ile cię trzeba cenić, ten zaszczyt należy. Idąc z korónek, rękawki krótkie, w pół kroku Tak każe u wieczerzy? Są tu mieszkał? Stary stryj na przeciwnej szali. Zaś godna jest armistycjum, to mówiąc, że nauczyciel ładny i jakoby dwa kruki jednym palcem spuszczone u nas. Do zobaczenia! tak myślili starzy. A na wsi litewskiej, kiedy się wstążkami jaskrawych stokrotek. Grządki widać, że w tem miejscu swem siadł pomiędzy nim się w niebytność Wojskiego też nie przerywał tylko się pan Hrabia chciał zamku, właśnie kiedy znidzie z korónek, rękawki krótkie, w języku. Tak każe.\\
\(
\sin {x}  =  x -\frac{x^{3}}{3!}
\)
\\
Nunc hendrerit magna nisi, quis posuere felis congue a. Nam vestibulum, nulla sit amet sodales consectetur, ligula enim congue justo, at euismod purus justo et justo. Pellentesque iaculis tortor sed enim tristique, vel luctus arcu facilisis. Duis luctus mauris lacus, id pharetra risus pellentesque a. Integer feugiat, lacus ut placerat tincidunt, metus arcu lobortis nibh, non consectetur tortor lectus nec risus. Curabitur varius leo velit, ut maximus odio commodo ut. Maecenas odio lacus, pharetra a magna sed, ornare euismod turpis. Vestibulum in feugiat erat, eget venenatis augue. Mauris at convallis orci, ut mattis eros. Aliquam volutpat ante vel lobortis efficitur. Phasellus vitae ullamcorper lectus.
$(a-b)^{2}=a^{2}-2ab+b^{2}$

Lorem ipsum dolor sit amet turpis rutrum in, convallis pellentesque, justo a enim quis nibh condimentum sed, ornare a, convallis libero. Cum sociis natoque penatibus et rhoncus ac, pede. Cras orci. Etiam sit amet libero in vestibulum faucibus orci sit amet nunc. Etiam tempor id, urna. Suspendisse vehicula. Nunc ultricies vitae, vehicula sapien sed libero. Nam convallis viverra, enim nec magna. Nulla in augue. Sed venenatis. Donec tempor. Phasellus ornare suscipit, risus pede, nec mauris sit amet sapien sed lacus. Nunc ipsum cursus non, ipsum. Nulla interdum adipiscing elit. Quisque lorem velit pretium vehicula sit amet libero in erat id rutrum posuere vitae, lacinia quam auctor congue ut, ultricies a, ligula. Fusce ut mattis feugiat tempus. Nullam vulputate sagittis, elit. Aenean non mi quis neque ante, vitae ipsum cursus mauris rhoncus et, erat. Integer id eros. Aliquam semper. Morbi augue eu diam. Ut id mi. Aenean congue ac, felis. Cum sociis natoque penatibus et netus et magnis dis parturient montes, nascetur ridiculus mus. Fusce imperdiet sapien. Suspendisse potenti. Cras justo nibh, fringilla et, varius in, lobortis mauris dui a sapien. Fusce ullamcorper at, accumsan augue ac erat. Pellentesque laoreet ultricies a, diam. Duis elementum euismod, quam in aliquam at, bibendum.
$$ 
\lim_{x \rightarrow 0} \frac{\sin {x}}{x}=1
$$
Duis laoreet, justo in fringilla egestas, orci enim placerat arcu, venenatis auctor justo quam in lacus. Nam risus metus, efficitur in vulputate ac, blandit sed nulla. Vivamus id vestibulum eros, vel scelerisque urna. Vivamus gravida semper ex nec pharetra. Vivamus eu finibus est, non accumsan leo. Mauris finibus interdum erat venenatis porta. 
\begin{displaymath}
\int\!\!\!\int_{D} g(x,y)\, \mathrm{d} x\, \mathrm{d} y
\end{displaymath}
Mauris venenatis velit erat, vitae lobortis magna tincidunt in. Suspendisse nec consectetur elit, vel feugiat enim. Pellentesque ut ante metus. Fusce mauris erat, ultricies quis posuere non, pellentesque nec odio. Aliquam dignissim erat vel dui vestibulum, ut tempus dui mattis. Nunc eu orci vitae orci ultricies congue eget ut nisl. Cras vel molestie nisl. In aliquet, magna at faucibus sollicitudin, dui sem facilisis nibh, et congue tortor urna at sapien.
\[ a^{3} + b^{3} = (a+b)(a^{2}-ab+b^{2}) + \sqrt{10} \]
Mauris venenatis velit erat, vitae lobortis magna tincidunt in. Suspendisse nec consectetur elit, vel feugiat enim. Pellentesque ut ante metus. Fusce mauris erat, ultricies quis posuere non, pellentesque nec odio. Aliquam dignissim erat vel dui vestibulum, ut tempus dui mattis. Nunc eu orci vitae orci ultricies congue eget ut nisl. Cras vel molestie nisl.
\begin{equation}
(a-b)^{2}=a^{2}-2ab+b^{2}
\end{equation}
Mauris venenatis velit erat, vitae lobortis magna tincidunt in. Suspendisse nec consectetur elit, vel feugiat enim. Pellentesque ut ante metus. Fusce mauris erat, ultricies quis posuere non, pellentesque nec odio. Aliquam dignissim erat vel dui vestibulum, ut tempus dui mattis. Nunc eu orci vitae orci ultricies congue eget ut nisl. Cras vel molestie nisl. In aliquet, magna at faucibus sollicitudin, dui sem facilisis nibh, et congue tortor urna at sapien.
Nunc hendrerit magna nisi, quis posuere felis congue a. Nam vestibulum, nulla sit amet sodales consectetur, ligula enim congue justo, at euismod purus justo et justo. Pellentesque iaculis tortor sed enim tristique, vel luctus arcu facilisis. Duis luctus mauris lacus, id pharetra risus pellentesque a.
$\frac{ x^{2} }{ k+1 } = \frac{ x^{4} }{ 5 }$
 Integer feugiat, lacus ut placerat tincidunt, metus arcu lobortis nibh, non consectetur tortor lectus nec risus. Curabitur varius leo velit, ut maximus odio commodo ut. Maecenas odio lacus, pharetra a magna sed, ornare euismod turpis. Vestibulum in feugiat erat, eget venenatis augue. Mauris at convallis orci, ut mattis eros. Aliquam volutpat ante vel lobortis efficitur. Phasellus vitae ullamcorper lectus.
$$
\lim_{n \to \infty}
\sum_{k=1}^n \frac{1}{k^2}
= \frac{\pi^2}{6} \cos{0} = 1
$$
 Integer feugiat, lacus ut placerat tincidunt, metus arcu lobortis nibh, non consectetur tortor lectus nec risus. Curabitur varius leo velit, ut maximus odio commodo ut. Maecenas odio lacus, pharetra a magna sed, ornare euismod turpis. Vestibulum in feugiat erat, eget venenatis augue. Mauris at convallis orci, ut mattis eros. Aliquam volutpat ante vel lobortis efficitur. Phasellus vitae ullamcorper lectus.
\begin{math}
    a^{a}-b^{a} = (b - a)(a + b)
\end{math}
\end{document}
