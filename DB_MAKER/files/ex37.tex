\documentclass{article}
\usepackage[utf8]{inputenc}
\usepackage[MeX,plmath]{polski} 
\usepackage{amsmath}
\usepackage{amsfonts}

\begin{document}
\title{Twierdzenie Abela o szeregach potęgowych}
\maketitle

\section*{Wprowadzenie}
Twierdzenie Abela o szeregach potęgowych – twierdzenie analizy zespolonej wiążące zbieżność szeregu potęgowego w punkcie brzegu koła zbieżności ze zbieżnością funkcji reprezentowanej przez szereg wewnątrz koła dla argumentów zbieżnych do tego punktu po pewnej drodze udowodnione przez norweskiego matematyka Nielsa Henrika Abela. 


\section*{Sformułowanie}

Niech an będzie ciągiem zespolonym: \((a_{n})_{n\in \mathbb {N} }\in \mathbb {C} ^{\mathbb {N} }\). Jeżeli szereg $\sum _{n=0}^{\infty }a_{n}$ jest zbieżny oraz funkcja zespolona określona w kole jednostkowym
$$f:\{z:|z|<1\}\to \mathbb {C}$$
jest dana wzorem \[f(z)=\sum _{n=0}^{\infty }a_{n}z^{n}\] 
to wówczas 
$$\sum _{n=0}^{\infty }a_{n}=\lim _{z\to 1}f(z)$$ gdy z dąży do 1 po drodze zawartej pomiędzy dwiema cięciwami koła zbieżności wychodzącymi z punktu 1. 

Uwagi: Przykładem takiej drogi może być odcinek otwarty (0,1). Przypadek dowolnego skończonego promienia zbieżności i punktu z jego brzegu może być sprowadzony do promienia 1 i punktu 1. 

\section*{Dowód}
Oznaczając przez sn sumy częściowe szeregu $ \sum _{n=0}^{\infty }a_{n}$, a przez s jego sumę i korzystając z przekształcenia Abela można zapisać: 

\begin{displaymath}
	f(z)=\sum _{n=0}^{\infty }a_{n}z^{n}=s_{0}+\sum _{n=1}^{\infty }(s_{n}-s_{n-1})z^{n}=\sum _{n=0}^{\infty }s_{n}(z^{n}-z^{n+1})=(1-z)\sum _{n=0}^{\infty }s_{n}z^{n}
\end{displaymath}

Zgodnie ze wzorem na granicę szeregu geometrycznego:

\begin{equation*}
	s=(1-z)\sum _{n=0}^{\infty }sz^{n}
\end{equation*}

a zatem: 

\[f(z)-s=(1-z)\sum _{n=0}^{\infty }(s_{n}-s)z^{n}\]

Ze zbieżności szeregu wynika, że można dobrać takie N, by dla każdego n większe od N(sn - s) było dostatecznie małe (mniejsze od ustalonego epsilon większe od 0). 

Korzystamy z potęgi punktu 1 względem okręgu o środku 0 przechodzącego przez z dla prostych przechodzących przez z (wtedy jeden z odcinków ma długość |1-z| i 0 (wtedy jeden z odcinków ma długość 1-|z|).

Wnioskujemy, że jeśli z leży pomiędzy pewnymi cięciwami (można zakładać, że cięciwy są symetryczne względem (0,1), bo zmiana cięciwy pod mniejszym kątem na symetryczną do drugiej zwiększa obszar zawarty między nimi), a |z|>r, gdzie r to promień okręgu o środku 0 stycznego do obu cięciw (dla z dostatecznie bliskich 1 można tak zakładać), to zachodzi nierówność: 

\begin{equation}
	{\frac {|1-z|}{1-|z|}}<{\frac {2}{l}}
\end{equation}

gdzie l jest długością odcinka pomiędzy 1 a punktem styczności cięciwy.  



\end{document}