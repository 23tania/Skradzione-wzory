\documentclass{article}
\usepackage[utf8]{inputenc}

\begin{document}
\title{Pierwiastki rzeczywiste równania kanonicznego o współczynnikach rzeczywistyc}
\maketitle

\section{Pierwiastki rzeczywiste równania kanonicznego o współczynnikach rzeczywistyc}
W oparciu o dyskusję w poprzedniej sekcji możemy podać gotowe wzory na pierwiastki rzeczywiste równań w postaci kanonicznej. Rozważamy następujące równanie: 
$${ y^{3}+py+q=0.}$$
gdzie współczynniki p , q są liczbami rzeczywistymi. Określmy jego wyróżnik jako $${ \Delta =\left({\frac {p}{3}}\right)^{3}+\left({\frac {q}{2}}\right)^{2}.}$$
Przypadek 1 delta > 0:
$$y_1=\sqrt[3]{-\frac{q}{2}-\sqrt{\Delta}}+\sqrt[3]{-\frac{q}{2}+\sqrt{\Delta}}$$
jest jedynym pierwiastkiem rzeczywistym równania (2). 
Przypadek 2   delta = 0 : 
Wówczas równanie (2) ma co najwyżej dwa rozwiązania w liczbach rzeczywistych: 
$y_1=\sqrt[3]{\frac{q}{2}}$ oraz $y_2=-2\sqrt[3]{\frac{q}{2}}.$
Przypadek 3   delta < 0 :
W tym przypadku równanie (2) ma trzy różne pierwiastki rzeczywiste. Aby wyznaczyć i opisać te pierwiastki, używamy funkcji trygonometrycznych i postaci trygonometrycznej liczb zespolonych.

Ponieważ $\Delta=\frac{q^2}{4}+\frac{p^3}{27}<0,$ to $-\frac{p^3}{27}>\frac{q^2}{4}\geq 0,$ stąd
$${ \left|{\frac {-{\frac {q}{2}}}{\sqrt {\frac {-p^{3}}{27}}}}\right|<1.}$$
oraz wybrać liczbę
$$v_0=\sqrt[3]{r}\left(\cos\frac{\varphi}{3}+i\sin\frac{\varphi}{3}\right)$$
$$u_0=\sqrt[3]{r}\left(\cos\frac{\varphi}{3}-i\sin\frac{\varphi}{3}\right)$$
\end{document}