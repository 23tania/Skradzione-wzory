\documentclass{article}
\usepackage[utf8]{inputenc}
\usepackage[MeX,plmath]{polski} 
\usepackage{amsmath}
\usepackage{amsfonts}

\begin{document}
\title{Funkcja Carmichaela}
\maketitle

\section*{Wprowadzenie}
Funkcja lambda Carmichaela - funkcja określona dla dodatnich liczb całkowitych, której wartością dla danej liczby n jest najmniejsza liczba, taka, że podniesiona do jej potęgi liczba względnie pierwsza z n przystaje do 1 mod n, przy czym lambda(0). 
$$ \forall _{k<n}\left[{\mbox{NWD}}(k,n)=1\Rightarrow k^{\lambda (n)}\operatorname {mod} n=1\right] $$
gdzie NWD to największy wspólny dzielnik, a ,,mod n'' – reszta z dzielenia przez n.

\section*{Definicja formalna}
Ścisła definicja funkcji Carmichaela jest taka, że dla danej liczby n, lambda(n) to najmniejsza taka liczba, że: 
\[ \forall _{k<n,\ NWD(k,n)=1}\ k^{\lambda (n)}\operatorname {mod} n=1 \]
gdzie NWD to największy wspólny dzielnik, a ,,mod n'' – reszta z dzielenia przez n.

Wychodząc od pojęcia grupy, pojęcie funkcji Carmichaela można wprowadzić dużo naturalniej. Mianowicie, jeżeli rozważymy multiplikatywną grupę klas reszt modulo n z działaniem mnożenia modulo n to: 
\begin{equation*}
\lambda (n)=\min\{k:\forall _{x\in Z_{n}^{*}}\ x^{k}\equiv 1\}
\end{equation*}
przy czym powyższe potęgowanie należy rozumieć jako składanie działania z grupy. 

\section*{Własności}
\subsection*{Ścisły wzór}
Ścisły wzór na funkcję lambda jest następujący (w poniższym wzorze pi to dla różnych indeksów różne liczby pierwsze, a alfai – liczby naturalne): 
\begin{displaymath}
\lambda (n)=\left\{{\begin{array}{cl}\phi (n)&n=p_{i}^{\alpha _{i}},\ p_{i}>2\lor \alpha _{i}<3\\{\frac {\phi (n)}{2}}&n=2^{\alpha _{i}},\ \alpha _{i}>2\\NWW{\big (}\lambda (p_{1}^{\alpha _{1}}),\dots ,\lambda (p_{k}^{\alpha _{k}}){\big )}&n=\prod _{i=1}^{k}p_{i}^{\alpha _{i}}\end{array}}\right.
\end{displaymath}
przy czym NWW to Najmniejsza wspólna wielokrotność. 

\subsection*{Oszacowania}
Dla dowolnej liczby naturalnej k zachodzi oszacowanie górne:
$$ \lambda (k)\leq \phi (k) $$

Natomiast zachodzi również nietrywialne oszacowanie górne dla nieskończenie wielu n:
$$ \lambda (n)<\ln(n)^{3,24\log _{3}n} $$
i oszacowanie dolne dla dostatecznie dużych n:
$$ \lambda (n)>\ln(n)^{1,44\log _{3}n} $$

\subsection*{Wartości dla potęg liczby 2}
Dla potęg liczby dwa zachodzą następujące równości: 
\[ \lambda (2^{n})=\phi (2^{n}) ~~{\text{dla}}~~ 0\leq n\leq 2 \]
\[ \lambda (2^{n})=2^{n-2}={\frac {\phi (2^{n})}{2}} ~~{\text{dla}}~~ n\geq 3 \]

\subsection*{Wartość dla liczb pierwszych}
Jeżeli p - liczba pierwsza to zachodzi:
\begin{displaymath}
\lambda (p)=\phi (p)=p-1
\end{displaymath}

\end{document}
